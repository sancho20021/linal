\documentclass[12pt]{article}
\usepackage[utf8]{inputenc}
\usepackage[russian]{babel}
\usepackage{amssymb}
\usepackage{systeme}
\usepackage{amsmath}
\usepackage{amsthm}
\usepackage{graphicx}
\usepackage{mathtools}
\usepackage{bbold}
\usepackage{enumitem}
\usepackage{collectbox}
\usepackage{multicol}
\usepackage[margin=0.5in]{geometry}
\usepackage{tabularx}
\usepackage[scr=boondoxo,scrscaled=1.05]{mathalfa}
\usepackage{stackengine}
\usepackage{wrapfig}
\usepackage{relsize}
\usepackage{latexsym}
\usepackage{caption}
\usepackage{rotating}
\usepackage{cancel}%strickeout%
\usepackage{ulem}

%\usepackage{pdfpages}

\title{Оформленное 2 задание, КР по квадратичным формам}
\author{Пак Александр}

%----ENVIRONMENTS--------------%
\newtheorem{theorem}{Теорема}[subsection]
\renewcommand{\thetheorem}{\arabic{theorem}}
\newtheorem{lemma}{Лемма}[subsection]
\renewcommand{\thelemma}{\arabic{lemma}}
\newtheorem{defin}{Определение}[subsection]
\renewcommand{\thedefin}{\arabic{defin}}
\newtheorem*{stat}{Утверждение}
\newtheorem{corollary}{Следствие}[theorem]
\renewcommand{\thecorollary}{\arabic{corollary}}

\newenvironment{mylist}{\begin{enumerate}[noitemsep, nolistsep]}{\end{enumerate}}

\theoremstyle{remark}
\newtheorem*{remark}{Замечание}

\theoremstyle{definition}
\newtheorem*{examples}{Примеры}

\newenvironment{Figure}
{\par\medskip\noindent\minipage{\linewidth}}
{\endminipage\par\medskip}

%-------------------------------%

%------COMMANDS-----------------%
\newcommand{\N}{\numberset{N}}
\newcommand{\Q}{\mathbb Q}
\newcommand{\R}{\mathbb R}
\newcommand{\Z}{\mathbb Z}
\newcommand{\0}{\mathbb{0}}
\newcommand{\mybox}{
	\collectbox{
		\setlength{\fboxsep}{1pt}
		\fbox{\BOXCONTENT}
	}
}
\newcommand{\E}{\mathcal{E}}
\newcommand{\A}{\mathcal{A}}
\newcommand{\B}{\mathcal{B}}
\newcommand{\D}{\mathcal{D}}
\newcommand{\J}{\mathcal{J}}
\let\vec\overline
\newcommand{\p}{\mathcal{P}}
\newcommand{\pu}{\sqsupset}
\def\dunderline#1{\underline{\underline{#1}}}
\newcommand{\ns}{\\[1pt]}
\newcommand{\nm}{\\[2pt]}
\newcommand{\nb}{\\[4pt]}
\newcommand{\nl}{\\[8pt]}
\newcommand{\n}{\\[6pt]}
\def\slide#1{\phantom{}\hspace{#1}}
%-------------------------------%

\setlength{\parindent}{0ex}
\linespread{1.2}


\begin{document}
	\pagenumbering{gobble}
	\maketitle
	\newpage
	\pagenumbering{arabic}
	\section{Привести кв. форму $f$ к кв. форме $g$, \\используя метод Лагранжа:}
	\[f(x) = x_1 x_2 + x_2 x_3 + x_3 x_4 + x_4 x_1\]
	\[g(x) = x_2^2 - 3 x_4^2 - 2x_2 x_3 + 2 x_2 x_4 - 6 x_3 x_4\]
	\subsection{Приведем $f$ к каноническому виду, найдем $Q_f$}
	$f(x) = x_1 x_2 + x_2 x_3 + x_3 x_4 + x_1 x_4$\n 
	Сделаем замену\n 
	$y_1 + y_2 = x_1\\
	y_1 - y_2 = x_2\\
	y_3 = x_3\\
	y_4 = x_4$\n 
	$\tilde{f}(y) = y_1^2 - y_2^2 + y_1 y_3 - y_2 y_3 + y_3 y_4 + y_1 y_4 + y_2 y_4$\n
	Занесем под один квадрат все множители с $y_1$\n 
	$\tilde{f}(y) = (y_1 + \frac{y_3}{2} + \frac{y_4}{2})^2 - \frac{y_3^2}{4} - \frac{y_4^2}{4} - y_2^2 - y_2y_3 + \frac{y_3 y_4}{2} + y_2 y_4$\n
	Занесем под квадрат все множители с $y_2\n 
	\tilde{f}(y) = (y_1 + \frac{y_3}{2} + \frac{y_4}{2})^2 - (y_2 + \frac{y_3}{2} - \frac{y_4}{2})^2$\n
	Сделаем замену\n 
	$t_1 = y_1 + \frac{y_3}{2} + \frac{y_4}{2}\\
	t_2 = y_2 + \frac{y_3}{2} - \frac{y_4}{2}\\
	t_3 = y_3\\
	t_4 = y_4\n
	\tilde{\tilde{f}}(t) = t_1^2 - t_2^2 \slide{30px} \sigma = (1, 1, 2)$\n
	\underline{Выразим $x$ через $t$}\n 
	$
	\left.
	\begin{array}{rcl}
		y_1 + \frac{y_3}{2} + \frac{y_4}{2} &=& t_1\\
		y_2 + \frac{y_3}{2} - \frac{y_4}{2} &=& t_2\\
		y_3 &=& t_3\\
		y_4 &=& t_4
	\end{array}\right\} \Rightarrow \left.\begin{array}{rcl}
		y_1 &=& t_1 - \frac{t_3}{2} - \frac{t_4}{2}\\
		y_2 &=& t_2 - \frac{t_3}{2} + \frac{t_4}{2}\\
		y_3 &=& t_3\\
		y_4 &=& t_4
	\end{array}\right\}\n 
	\left.
	\begin{array}{rcl}
		x_1 &=& t_1 + t_2 - t_3\\
		x_2 &=& t_1 - t_2 -t_4\\
		x_3 &=& t_3\\
		x_4 &=& t_4
	\end{array}\right\} \Rightarrow Q_f = \begin{pmatrix}
		1 & 1 & -1 & 0\\
		1 & -1 & 0 & -1\\
		0 & 0 & 1 & 0\\
		0 & 0 & 0 & 1
	\end{pmatrix}$
	\subsection{Приведем $g$ к каноническому виду, найдем $Q_g^{-1}$}
	$g(x) = x_2^2 - 3 x_4^2 - 2x_2 x_3 + 2 x_2 x_4 - 6 x_3 x_4$\n 
	Занесем под квадрат все множители с $x_2$\n
	$g(x) = (x_2 - x_3 + x_4)^2 - x_3^2 - x_4^2 + 2x_3 x_4 - 3x_4^2 - 6x_3 x_4$\n
	Занесем под квадрат все множители с $x_3$\n 
	$g(x) = (x_2 - x_3 + x_4)^2 - (x_3 + 2 x_4)^2$\n 
	Сделаем замену\n
	$t_1 = x_2 - x_3 + x_4\\
	t_2 = x_3 + 2x_4\\
	t_3 = x_1\\
	t_4 = x_4$\n 
	$\tilde{g}(t) = t_1^2 - t_2^2 \slide{30px} \sigma = (1, 1, 2)\n$
	Получили $Q^{-1}_g = \begin{pmatrix}
		0 & 1 & -1 & 1\\
		0 & 0 & 1 & 2\\
		1 & 0 & 0 & 0\\
		0 & 0 & 0 & 1
	\end{pmatrix}$
	\subsection{Найдем $Q$}
	$\tilde{\tilde{f}}(t) = \tilde{g}(t)\n 
	Q = Q_f \cdot Q^{-1}_g = \begin{pmatrix}
	1 & 1 & -1 & 0\\
	1 & -1 & 0 & -1\\
	0 & 0 & 1 & 0\\
	0 & 0 & 0 & 1
	\end{pmatrix} \cdot \begin{pmatrix}
	0 & 1 & -1 & 1\\
	0 & 0 & 1 & 2\\
	1 & 0 & 0 & 0\\
	0 & 0 & 0 & 1
	\end{pmatrix} = \begin{pmatrix}
		-1 & 1 & 0 & 3\\0 & 1 & -2 & -2\\
		1 & 0 & 0 & 0\\
		0 & 0 & 0 & 1
	\end{pmatrix}$\n 
	\underline{Сделаем проверку}\n
	$F = \begin{pmatrix}
		0 & \frac{1}{2} & 0 & \frac{1}{2}\\
		\frac{1}{2} & 0 & \frac{1}{2} & 0\\
		0 & \frac{1}{2} & 0 & \frac{1}{2}\\
		\frac{1}{2} & 0 & \frac{1}{2} & 0
	\end{pmatrix}\; \; \; \; G = \begin{pmatrix}
		0 & 0 & 0 & 0\\
		0 & 1 & -1 & 1\\
		0 & -1 & 0 & -3\\
		0 & 1 & -3 & -3
	\end{pmatrix}\n 
	G = Q^T F Q$ --- верно.
\end{document}