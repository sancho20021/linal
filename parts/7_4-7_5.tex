\documentclass[../main.tex]{subfiles}
\begin{document}
	\subsection{Собственные числа и собственные вектора линейного оператора.}
	$\A\in End(v) \ \ \ \ V$ линейное пространство над $K$
	\begin{defin}
		$\lambda\in K \text{ -- \textbf{собственное число} (c.ч.) линейного оператора }\A$, если $\\
		\exists \ \boxed {v\in V \neq \0}$, который называется \textbf{собственным вектором} (с.в.), такой что $\boxed{\A v = \lambda v}$
	\end{defin}
	Пусть $v: \A v = \lambda v \Leftrightarrow (\A-\lambda\E)v = 0 \Leftrightarrow v\in Ker(\A-\lambda\E)$
	\begin{defin}
		$V_\lambda = Ker(\A-\lambda\E) = \{\text{с.в. }v \text{ и }\0\}$ называется собственным подпространством.\\
		$\boxed{\gamma(\lambda):= \dim V_\lambda} $ -- геометрическая кратность с.ч.
	\end{defin}
	$\gamma\geq 1\\
	V_\lambda$ и $\gamma(\lambda)$ -- инварианты относительно выбора базиса.\\
	$v\in V_\lambda \ \ \A v = \lambda v\overset{?}{\in} V_\lambda\\
	\A(\lambda v) = \lambda\A v = \lambda^2 v = \lambda(\lambda v)$
	\begin{examples}
		\ \\
		\begin{mylist}
			\item 
			$\A$ -- оператор подобия:\\
			$\A v = \mu \cdot v \ \ \ \mu \in K\\
			\mu$ с.ч. $\ \ V_\lambda = V$
			\item 
			$\A$ -- оператор поворота на плоскости на угол $\alpha$\\
			\begin{minipage}{0.2\textwidth}
				\includegraphics[width=100px]{11}
			\end{minipage}
			\begin{minipage}{0.8\textwidth}
				$\alpha \neq \pi k \Rightarrow$ нет с.в.
			\end{minipage} \\
			\item 
			Пусть $\lambda$ с.ч.$=0 \ \ \ \A v = \0 \ $с.в. $\neq \0 \ \ \Leftrightarrow\\
			\Leftrightarrow Ker\A$ нетривиально $\Leftrightarrow \A$ не автоморфизм $\Leftrightarrow \A$ необратимо $\Leftrightarrow det\A = 0$ 
			\item $\A: V\rightarrow V\\
			v_1\ldots v_n$ базис, т.ч. $A = \begin{pmatrix}
			\lambda_1 & \ldots & 0\\
			\ldots & \ldots & \ldots\\
			0 & \ldots & \lambda_n
			\end{pmatrix} = diag(\lambda_1\ldots\lambda_n) = \Lambda$\\
			Базис состоит из с.в. отвечающих с.ч. $\lambda_1 \ldots \lambda_n\\
			\A v_i = \lambda_i v_i \ \ \ A_i = \begin{pmatrix}
			0\\0\\\lambda_i\\\vdots\\0
			\end{pmatrix}\\
			\lambda$ -- с.ч. $v$ с.в. $\neq \0 \Leftrightarrow Ker(\A-\lambda\E)$ нетривиально $\Leftrightarrow det(\A - \lambda\E) = 0$
		\end{mylist}
	\end{examples}
	\begin{defin}
		$\chi_\A (t) = det(\A-t\E)$ -- характеристический многочлен оператора $\A, t\in K$
	\end{defin}
	
	$V e_1\ldots e_n $ базис $\ \ \A\leftrightarrow A\\
	\chi_\A(t) = det(\A-t\E) = det(A-tE)$ т.к. $det$ оператора инвариантен относительно выбора базиса.\\
	$\chi_\A(t) = det(A-tE) = \left|\begin{matrix}
	(a_{11}-t) & a_{12} & \ldots & a_{1n}\\
	a_{21} & (a_{22}-t) & \ldots & \ldots\\
	\ldots & \ldots & \ldots & \ldots\\
	a_{n1} & \ldots & \ldots & (a_{nn}-t)
	\end{matrix}\right| = \\
	= (-1)^n t^n + (-1)^{n-1}(\underset{tr A = tr\A}{a_{11} + \ldots + a_{nn}})t^{n-1} + \ldots + \underset{det\A}{detA}$\\\\
	По теореме Виета: $det\A = \underset{\text{корни }\chi_\A(t)}{\lambda_1 \ldots \lambda_n}\\\\
	\underline{\underline{\lambda\in K}}$ с.ч. $\Leftrightarrow \chi_\A(\lambda) = 0 \ \ (\underline{\underline{\lambda\in K}})$\\
	$\lambda$ корень характеристического многочлена.\\
	$k = \mathbb{C} \Rightarrow n$ с.ч. с учетом кратности корней характеристического многочлена.\\
	$k = \R \Rightarrow$ только вещественные корни $\chi_A$ будут с.ч.\\
	$\chi_\A(t) = (-1)^n \prod\limits_{\lambda \text{ корень}} (t-\lambda)^{\alpha(\lambda)}\\
	\alpha(\lambda)$ называется алгебраической кратностью с.ч. $\lambda$ (если $\lambda \in K$)
	\begin{defin}
		Множество всех с.ч. с учетом алгебраической кратности называется \textbf{спектром} линейного оператора. ($\lambda, \alpha(\lambda)$)\\
		Спектр -- простой, если все с.ч. попарно-различны.\\
		$\alpha(\lambda) = 1 \ \forall \ \lambda$
	\end{defin}
	\textbf{Немножко про алгебраическую кратность}\\
	$f(t) = a_nt^n + a_{n-1}t^{n-1} + \ldots + a_1t + a_0 = a_n\prod\limits_{a \text{--корень}}(t-a)^{m_a}\\
	a\text{--корень } f \Leftrightarrow f(a) = 0 \Leftrightarrow f\ \vdots\ (t-a)\\
	a$ -- корень $f$ \textbf{кратности} $m \Leftrightarrow \begin{matrix}
	f \mid (t-a)^m\\
	f \nmid (t-a)^{m+1}
	\end{matrix}\\
	\Leftrightarrow f(t) = (t-a)^m g(t)\\
	a_0$ -- произведение всех корней с учетом кратности = $(-1)^n \prod a \ \ \ \ \ a$--корень с учетом кратности
	\\\\
	$\det\A = \lambda_1 \ldots \lambda_n\\
	(-1)^n t^n + \ldots = (-1)^n (t-a_1)(t-a_2)\ldots(t-a_n)\\
	\chi_\A (t) = (-1)^n t^n + \ldots = (-1)^n (t-\lambda_1) \ldots (t-\lambda_n)\\
	\det\A = \lambda_1\ldots\lambda_n = 0 \Leftrightarrow \lambda = 0 $ с.ч.
	\begin{examples}
		$\A$ -- поворот на угол $\alpha$\\
		\begin{minipage}{0.3\textwidth}
			\includegraphics[width=\textwidth]{11}
		\end{minipage}
		\begin{minipage}{0.7\textwidth}
			$\vec{i}\ \vec{j}\\
			\A \leftrightarrow A = \begin{pmatrix}
			\cos\alpha & -\sin\alpha\\
			\sin\alpha & \cos\alpha
			\end{pmatrix}$
		\end{minipage}
		$\chi_\A(t) = det \begin{pmatrix}
		\cos\alpha-t & -\sin\alpha\\
		\sin\alpha & \cos\alpha-t
		\end{pmatrix} = \\
		\cos^2\alpha - 2\cos\alpha t + t^2 + sin^2\alpha = t^2 - 2\cos\alpha t + 1\\
		D = 4\cos^2\alpha - 4 < 0 \ \ \ \alpha\neq \pi k$\\
		нет вещ. корней $\Rightarrow$ нет с.ч.\\
		$K = \R$
	\end{examples}
	\begin{theorem}
		$\lambda$ с.ч. $\A \Rightarrow \boxed{1\leq \gamma(\lambda) \leq \alpha (\lambda)}$
	\end{theorem}
	\begin{proof}
		Пусть $\gamma(\lambda) = k = dimV_\lambda = span(\underset{\text{базис}}{v_1\ldots v_k})\\
		V_\lambda$ инвариантно относительно $\A\Rightarrow \exists $ базис: матрица оператора будет иметь вид:\\
		\textit{(инвариантное линейное подпространство. Смотри Теорему пункта 7.3}\\
		$A = \begin{pmatrix}
		A^1 & \vline & A^2\\
		\hline
		0  & \vline & A^3	
		\end{pmatrix}
		= \left(\begin{array}{c|c}
		\begin{array}{cc}
		\lambda & 0\\
		0 & \lambda
		\end{array} & A^2\\
		\hline
		0  & A^3
		\end{array}\right) \ \ A^1_{k\times k}$\\
		
		Базис = $v_1\ldots v_k v_{k+1} \ldots v_n\\
		\A \underset{i = 1\ldots k}{v_i} \in V_\lambda = \lambda v_i \leftrightarrow A^1_i = \begin{pmatrix}0\\0\\\lambda\\0\\0\\0\end{pmatrix}$\\
		$\chi_\A(t) = det \left(
		\begin{array}{c|c}
		\begin{matrix}
		\lambda-t & 0\\0 & \lambda - t
		\end{matrix} & A^2\\
		\hline
		0 & A^3 - tE_{n-k}
		\end{array}\right) \underset{\text{св-ва } det}{=} \begin{vmatrix}
		\lambda - t & 0\\0 & \lambda-t
		\end{vmatrix}|A^3-t E_{n-k}| = (\lambda-t)^k\chi_{\A^3}(t)$\\
		Очевидно, $\lambda$ корень $\chi_\A(t)$ кратности не меньше, чем $k \Rightarrow\alpha(\lambda) \geq k = \gamma(\lambda)$
	\end{proof}
	\begin{theorem}
		$\lambda_1\ldots\lambda_m $ -- различные с.ч. $\A\\
		v_1\ldots v_m$ соответствующие им с.в. $\Rightarrow\\
		\Rightarrow v_1\ldots v_m$ линейно независимы.
	\end{theorem}
	\begin{proof} Метод математической индукции
		\begin{mylist}
			\item База. $m=1 \ \ \ \lambda_1 v_1$ с.в. -- линейно независимы, т.к. $v_1\neq \0$
			\item Индукционное предположение. Пусть верно для $m-1$
			\item Индукционный переход. Докажем, что верно для $m$\\
			От противного. Пусть $\lambda_1\ldots \lambda_m $ попарно различные с.ч. $\A$,\\
			а $v_1\ldots v_m$ линейно зависимы.\\
			Пусть $v_m = \sum\limits_{i=1}^{m}\alpha_i v_i\\
			\begin{array}{l}
			\A_{v_m} = \sum\limits_{i=1}^{m-1}\alpha_i \A_{v_i} = \sum\limits_{i=1}^{m-1} \alpha_i \lambda_i v_i\\
			
			||
			\\
			\lambda_m v_m = \sum\limits_{i=1}^{m-1} \alpha_i \lambda_m v_i
			\end{array}$\\
			$\sum\limits_{i=1}^{m-1}\alpha_i(\underset{\stackrel{0}{\nparallel}}{\lambda_i - \lambda_m})v_i = \0 \ \ v_i$ линейно независим по инд. предположению\\
			$\Leftrightarrow \alpha_i = 0 \ \ \forall \ i=1\ldots m-1 \Rightarrow\\
			\Rightarrow v_m = \0$ --- Противоречие, т.к. $v_m$ с.в. и значит не может быть $\0$
		\end{mylist}
	\end{proof}
	\begin{corollary}
		$\lambda_1\ldots\lambda_m$ различные с.ч. $\A
		\Rightarrow  V_{\lambda_1}\ldots V_{\lambda_m}$ дизъюнктны.
		$\left(\bigoplus\limits_{\stackrel{\lambda}{\text{с.ч.}}} V_\lambda\right)$ 
	\end{corollary}
	\begin{proof}
		$v_1 + \ldots +v_m = \0 \ \ v_i\ \in V_{\lambda_i}$\\
		Если хотя бы 1 слагаемое $\neq \0 \Rightarrow$ это слагаемое с.в. $\Rightarrow$ противоречие с линейной независимостью с.в., отвечающих различным с.ч. $\Rightarrow \forall \ i: v_i = \0 \Rightarrow$ дизъюнктны. 
	\end{proof}
	\begin{theorem}
		$V = \bigoplus\limits_{i=1}^m L_i \ \ L_i$ инвариантно относительно $\A\\
		\A_i = \A|_{L_i} : L_i \rightarrow L_i
		\Rightarrow \boxed{\chi_\A(t) = \prod_{i=1}^{m}\chi_{\A_i}(t)}$
	\end{theorem}
	\begin{proof}
		см. теорему - следствие п. 7.3\\
		Базис $V$ -- объединение базисов $L_i\\
		\A \leftrightarrow A = \begin{pmatrix}
		\boxed{A^1} & & 0\\
		& \boxed{A^2} & \\
		0 & & \boxed{A^m}
		\end{pmatrix}\\
		\A_i \leftrightarrow A^i \ \ \ \ \; A_{k_i\times k_i}\\
		\chi_\A(t) = |A-tE| \underset{\text{свойства }det}{=} |A^1-tE_{k_1}||A^2-tE_{k_2}|\ldots|A^m-tE_{k_m}| = \\
		\begin{matrix}
		\chi_{A^1}(t) & \chi_{A^2}(t) & \ldots & \chi_{A^m}(t)\\
		|| & || & & ||\\
		\A_1 & \A_2 & & \A^m
		\end{matrix}
		$
	\end{proof}
	Все свойства с.ч. и с.в. доказанные для оператора верны для числовых матриц пространств $\R^m, \mathbb{C}^m$.\\
	$A_{n\times n} \  \ \; \lambda$ с.ч. $A: \exists x\in \R^n \neq \0 \ \; \ Ax = \lambda x\\
	y = \underset{\stackrel{\uparrow}{\text{линейный оператор}}}{Ax}$
	\begin{examples}
		$A = \begin{pmatrix}
		4 & -5 & 2\\
		5 & -7 & 3\\
		6 & -9 & 4
		\end{pmatrix}$\\
		с.ч., с.в.?  $\alpha(\lambda), \gamma(\lambda)$?\\
		$\chi_{\A}(t) = \chi(t) = \begin{vmatrix}
		4-t & -5 & 2\\
		5 & -7-t & 3\\
		6 & -9 & 4-t
		\end{vmatrix} = \begin{vmatrix}
		4-t & 1-t & 2\\
		5 & 1-t & 3\\
		6 & 1-t & 4-t
		\end{vmatrix} = (1-t)\begin{vmatrix}
		4-t & 1 & 2\\
		5 & 1 & 3\\
		6 & 1 & 4-t
		\end{vmatrix}
		= (1-t)t^2\\
		t_1 = 0 \; \;\alpha(0) = 2\\
		t_2 = 1 \; \;\alpha(1) = 1\\
		\\
		V_\lambda = Ker(A-\lambda E) \ \ \ \ \ A = \begin{pmatrix}
		4 & -5 & 2\\
		5 & -7 & 3\\
		6 & -9 & 4
		\end{pmatrix}\\
		\lambda_1 = 0 \ \ \ \begin{pmatrix}
		4 & -5 & 2 \ \vline & 0\\
		5 & -7 & 3 \ \vline & 0\\
		6 & -9 & 4 \ \vline & 0
		\end{pmatrix} \sim \ldots \begin{pmatrix}
		x_1\\x_2\\x_3
		\end{pmatrix} = \alpha\begin{pmatrix}
		1\\2\\3
		\end{pmatrix}\ \ \ \alpha\in]R\\
		V_{\lambda_1} = 0 = span\begin{pmatrix}
		1\\2\\3
		\end{pmatrix}\\
		\gamma(0) = 1 < \alpha(0)\\
		\lambda_2 \ \ 1\leq\gamma \leq\alpha = 1\\
		\begin{pmatrix}
		3 & -5 & 2 \ \vline & 0\\
		5 & -8 & 3 \ \vline & 0\\
		6 & -9 & 3 \ \vline & 0
		\end{pmatrix} \sim \ldots \begin{pmatrix}
		x_1\\x_2\\x_3
		\end{pmatrix} = \alpha\begin{pmatrix}
		1\\1\\1
		\end{pmatrix}
		\ \ \alpha\in\R\\
		V_{\lambda_2} = span\begin{pmatrix}
		1\\1\\1
		\end{pmatrix}\\
		\gamma(1) = 1
		$
	\end{examples}
	\subsection{Оператор простой структуры. (о.п.с.) \\
		Проекторы. Спектральное разложение о.п.с.
		\\ Функция от матрицы.}
	\begin{defin}
		$\A\in End(V)\\$
		$\A$ называется о.п.с., если $\exists$ базис пространтсва $V$, т.ч. матрица оператора в этом базисе имеет диагональный вид $\Lambda = diag(\lambda_1\ldots\lambda_n) = \begin{pmatrix}
		\lambda_1 & 0\\ 0 & \lambda_n
		\end{pmatrix} \Leftrightarrow \exists$ базис $V$ из с.ч. $\A \Leftrightarrow V = \bigoplus\limits_{\lambda \text{с.ч. } \A} V_\lambda\\
		V = span(v_1\ldots v_n)$
	\end{defin}
	\begin{theorem}
		Пусть $\sum\limits_{\lambda \text{с.ч. }\A} \alpha(\lambda) = n = dim V\\
		\Leftrightarrow $все корни $\chi(t) \in K \Leftrightarrow$ все корни $\chi(t)$ являются с.ч. $\A$\\
		$\boxed{\A\text{о.п.с.}\Leftrightarrow \forall\text{с.ч.} \lambda \ \ \ 1 \leq \gamma(\lambda) = \alpha(\lambda)}$
	\end{theorem}
	\begin{proof}
		$\A$ о.п.с. $\Leftrightarrow V = \bigoplus\limits_{\lambda \text{с.ч.}}V_\lambda \Leftrightarrow \\
		\Leftrightarrow n = dim V = \sum\limits_{\lambda \text{с.ч.}}\gamma(\lambda) \underset{\nearrow}{=} \sum\limits_{\lambda\text{с.ч.}}\alpha(\lambda)\\
		1\leq \gamma(\lambda) \leq \alpha(\lambda) \ \ \ \ \; \; \; \nearrow\\
		\sum\limits_{\lambda\text{с.ч.}}\alpha(\lambda) = n\ \  \rightarrow \ \ \nearrow
		\Rightarrow \forall\lambda: \boxed{\gamma(\lambda) = \alpha(\lambda)}$
	\end{proof}
	\begin{corollary}
		$\sum\limits_{\lambda\text{с.ч.}}\alpha(\lambda) = n = dim V\\
		\A $о.п.с. $\Leftarrow $ спектр -- простой.\\
		($n$ попарно различных с.ч. $\forall \lambda \gamma(\lambda) = \alpha(\lambda) = 1$)
	\end{corollary}
	\begin{defin}
		$A_{n\times m}$ называется диагонализируемой, если $\exists$ невырожденная $T_{n\times n}$, т.ч.\\ $T^{-1}AT = \Lambda = diag(\lambda_1\ldots\lambda_n)$\\
		("$A$ подобна диагональной матрице")
	\end{defin}
	\begin{corollary}
		Если матрица $A_{n\times n}$ -- матрица некоторого о.п.с. $\A$, то она \textbf{диагонализируема}. И обратно, любая диагонализируемая матрица является матрицей о.п.с. в некотором базисе.
	\end{corollary}
	\begin{proof}\ \\
		\begin{tabular}{clccc}
			$\A$ & о.п.с. & $\Leftrightarrow$ & $\exists$ базис & $\underset{\text{с.в.}}{v_1\ldots v_n}$\\
			$\updownarrow$ & $\underset{\text{базис}}{(e_1\ldots e_n)}V$ & & & $\underset{\text{с.ч.}}{\lambda_1\ldots \lambda_n}$\\
			$A$ & & & & $\updownarrow$\\
			& & & &$\Lambda = \begin{pmatrix}
			\lambda_1 & 0\\ 0 & \lambda_n
			\end{pmatrix}$
		\end{tabular}\\
		$T = T_{e\rightarrow \mathscr{v}}$ невырожденная.\\
		$\Lambda = T^{-1} A T\\
		A = T \Lambda T^{-1}$
	\end{proof}
	$\boxed{
		\begin{array}{r}
		A \text{ диагонализируема }\Leftrightarrow \sum\limits_{\lambda\text{с.ч.}}\alpha(\lambda) = n\\
		\forall \ \lambda \text{ с.ч.} \ \gamma(\lambda) = \alpha (\lambda)
		\end{array}
	}$
	\begin{defin}\ \\
		$\begin{array}{ccc}
		V = \bigoplus\limits_{i=1}^m L_i & & \mathscr{p}_i : V\rightarrow L_i \subset V\\
		\nwarrow\Leftarrow & \Leftrightarrow & \Rightarrow\searrow\\
		\underset{\text{линейное подпр.}}{L_i \subset V} & & \forall v \in V \  \exists ! : v = \sum\limits_{i=1}^m v_i \in L_i
		\end{array}$\\
		$\boxed{\forall \ v\in V \ \ \ \p_i v \overset{def}{:=}v_i} \ \ \ \ \ \ \; i = 1\ldots m$
	\end{defin}
	\textbf{Оператор проектирования (проектор)}\\
	$\p_i \overset{?}{\in} End(V)\\
	\p_i (u + \lambda v) = u_i + \lambda v_i = \p_i u + \lambda \p_i v \ \ \ \Rightarrow \ \ \ \p_i $ линейный оператор.\\
	$u + \lambda V = \sum\limits_{i=1}^m u_i \in L_i + \lambda \sum\limits_{i=1}^m v_i \in L_i = \sum\limits_{i=1}^m (\underbrace{u_i + \lambda v_i}_{\in L_i})\\
	u_i = \p_i u \ \ \ \ v_i = \p_i v
	$
	\newpage
	\textbf{Свойства проекторов:}
	\begin{mylist}
		\item $\forall \ i \neq j \ \ \p_i \p_ij = \0$
		\item $
		\forall i: \p_i^2 = \p_i \ \ (\Rightarrow \forall k\in \mathbb{N} \o_i^k = \p_i)
		$
		\item $\sum\limits_{i=1}^m \p_i = \E$
		\item $Ker\p_i = \sum\limits_{j\neq i} L_j \ \ \forall i = 1\ldots m\\
		Im\p_i = L_i$
	\end{mylist}
	\begin{proof}\ \\
		\begin{mylist}
			\item $\forall v \in V \ \ \p_i\p_ij(v) = \p_i v_j \in L_j = \0 \Rightarrow \p_i\p_ij = \0$\\
			Т.к. $L_i$ дизъюнктны \\
			$v = v_1 + v_i + \underset{\text{Ед. образом}}{v_j} + \ldots + v_n\\
			v_j = v_j + \0$
			\item $\forall v\in V \ \ \p_i \underbracket{\p_i(v)}_{v_i\in L_i}=v_i = \p_i v$\\
			Т.к. верно $\forall v\in V$, то верно и для базиса $\Rightarrow$ операторы совпадают. $\p_i\p_i = \p_i$
			\item $\forall v \in V (\sum\limits_{i=1}^m \p_i)v = \sum\limits_{i=1}^m \p_i v = \sum\limits_{i=1}^m v_i = v = \E v \Rightarrow \ldots \Rightarrow \sum\limits_{i=1}^m = \E$
			\item 
			$\p_i(\underset{\mathlarger{= \sum\limits_{j \neq i} \underbrace{\p_i v_j}_{\0}}}{v_1 + \ldots + v_{i-1} + v_{i+1} + \ldots + v_m}) + \0\\
			\boxed{
				\begin{matrix}
				\sum\limits_{j\neq i} L_j \subset Ker\ \p_i\\
				\text{т.к. }v = \bigoplus\limits_{j\neq i} L_j \oplus L_i
				\end{matrix}
			} \Rightarrow Ker\ \p_i = \bigoplus\limits_{j\neq i} L_j\\\\
			Im \ \p_i = L_i \text{ по }def \ \ "\subset"$\\
			Верно "$\supset$" $\ \forall v_i \in L_i \leadsto v_i \in V = \p v_i = v_i$
		\end{mylist}
	\end{proof}
	\begin{stat}
		$\underset{i=1\ldots m}{\p_i \in End(V)} : V\rightarrow V$ и выполнены свойства 1, 3 $\Rightarrow\\
		\Rightarrow V = \bigoplus\limits_{i=1}^m Im\p_i $  (т.е. $\p_i$ проекторы на $L_i = Im\p_i$)
	\end{stat}
	\begin{proof}
		\ \\
		\begin{mylist}
			\item Если выполнены 1, 3, то верно 2\\
			$\p_i \p_i \overset{?}{=}\p_i\\
			\p_i = \p_i\E = p_i \sum\limits_{j=1}^m \p_j = \sum\limits_{j=1}^m \underset{\begin{matrix}
				\stackrel{||}{\stackrel{\0}{i\neq j}}
				\end{matrix}}{\p_i\p_j} = \p_i^2$
			\item 
			$v_1 + v_2 + \ldots + v_m = \0\\
			v_i \in Im\p_i \ $ дизъюнктно?\\
			$v_i = \p_i w_i \ w_i\in V\\
			v_i = \p_i w_i \underset{\uparrow}{=} \p_i (\underbrace{\sum\limits_{j=1}^m \underbracket{\p_j w_j}_{v_j}}_{=\0}) = \0\\
			\sum\limits_{j=1}^m \underbrace{\p_i ( p_j}_{=\0 \ i\neq j} w_j) = \p_i^2 w_i = \p_i w_i\\
			\forall \ v\in V \ \E v = v = \sum\limits_{j=1}^m \underset{\mathlarger{\stackrel{||}{v_j \in Im \p_j}}}{\p_j v}
			\Rightarrow v = \sum\limits_{j=1}^m Im \p_j$
		\end{mylist}
	\end{proof}
	\begin{theorem}[О спектральном разложении о.п.с.]
		$v = \bigoplus\limits_{\lambda\text{с.ч.}}V_\lambda \ \ \ \ \ \underset{\text{проекторы}}{\p_\lambda: 
			V \rightarrow V_\lambda}\\
		\A \text{ о.п.с.} \Leftrightarrow \A = \sum\limits_{\lambda\text{с.ч.}}\lambda\p_\lambda \leftarrow $спектральные проекторы
	\end{theorem}
	\begin{proof}
		\ \\
		\begin{mylist}
			\item 
			$\p_\lambda\p_\mu = \0$
			\item $\p^2_\lambda = \p_\lambda$
			\item $\sum\limits_{\lambda \text{с.ч.}}\p_\lambda = \E$
		\end{mylist}
		$\forall v\in V\\
		\A v \underset{\stackrel{\uparrow}{V = \bigoplus\limits_{\lambda} V_\lambda}}{=}
		\A(\sum\limits_{\lambda}v_\lambda\in V_\lambda) = \sum\limits_{\lambda\text{с.ч.}}\underbracket{\A v_\lambda}_{= \lambda v_\lambda} = \\
		\sum\limits_{\lambda\text{с.ч.}}\lambda v_\lambda = \sum\limits_{\lambda\text{с.ч.}}\lambda \p_\lambda v$\\
		Доказательство верно $\forall$ векторного про-ва $V$. В частности для базиса $\Rightarrow \boxed{\A = \sum\limits_{\lambda \text{с.ч.}}\lambda \p_\lambda}$
	\end{proof}
	\begin{corollary}
		\textbf{$A_{n\times n}$ диагонализируема} $\Leftrightarrow$ 
		\belowbaseline[-10pt]{$\begin{array}{c}
			\exists \ \underset{\text{проекторы}}{\p_{\lambda \ n\times n}} \ \ \ 1^\circ\ 2^\circ\ 3^\circ\\
			A = \sum\limits_{\lambda\text{с.ч.}}\lambda \p_\lambda
			\end{array}$}
	\end{corollary}
	\begin{examples}
		$A = $\belowbaseline[-10pt]{$\begin{pmatrix}
			7 &  -12 & 6\\
			10 & -19 & 10\\
			12 & -23 & 13
			\end{pmatrix}$}\\
		$\lambda_1 = 1 \ \alpha(\lambda_1) = \gamma(\alpha_1) = 2\\
		V_{\lambda_1} = span\begin{pmatrix}
		2\\1\\0
		\end{pmatrix}\begin{pmatrix}
		1\\0\\-1
		\end{pmatrix} = span(v_1, v_2)\\
		\lambda_2 = -1 \ \alpha(\lambda_2) = \gamma(\lambda
		_2) = 1\\
		V_{\lambda_2} = span \begin{pmatrix}
		3\\5\\6
		\end{pmatrix} = span \ V_3\\
		\Rightarrow \text{о.п.с. }V = V_{\lambda_1}\oplus V_{\lambda_2} = span(V_1, V_2, V_3)\\
		T_{e\rightarrow v} = \begin{pmatrix}
		2 & 1 & 3\\
		1 & 0 & 5\\
		0 & -1 & 6
		\end{pmatrix}\\
		T^{-1}AT = \begin{pmatrix}
		1 & 0 & 0\\
		0 & 1 & 0\\
		0 & 0 & -1
		\end{pmatrix} = \Lambda \ \ \ \boxed{AT = T\Lambda}\\
		\p_1: V\rightarrow V_{\lambda_1}\subset{V} \\
		\p_2: V\rightarrow V_{\lambda_2} \subset{V}\\
		\p_1' \text{ матрица }\p_1 \text{ в базисе } v = \begin{pmatrix}
		1 & 0 & 0\\
		0 & 1 & 0\\
		0 & 0 & 0
		\end{pmatrix}\\
		\p_1, \p_2 $-- матрицы проекторов в базисе $e$(канонич.)\\
		$\p_1 v_i = \left[\begin{array}{l}
		v_i, i = 1, 2\\
		\0, i = 3
		\end{array}\right.\\
		1^\circ \ 2^\circ \ 3^\circ\\
		\p_1' + \p_2' = E\\
		\p_1'\p_2' = \0 \ldots\\
		\p_2'$ матрица $\p_2$ в базисе $v = \begin{pmatrix}
		0 & 0 & 0\\
		0 & 0 & 0\\
		0 & 0 & 1
		\end{pmatrix}$
	\end{examples}
	\begin{examples}
		\belowbaseline[-10pt]{
			$A = \begin{pmatrix}
			7 & -12 & 6\\
			10 & -19 & 10\\
			12 & -24 & 13
			\end{pmatrix}$}\\
		$\p_i' = T^{-1}\p_i T \ \ \ \ \ i = 1, 2\\
		\p_i = T\p_i'T^{-1} \ \ \ \begin{array}{c}
		\p_1, \p_2 = \0\\
		\p_1^2 = \p_1
		\end{array}\\
		\p_1 = \begin{pmatrix}
		4 & -6 & 3\\
		6 & -9 & 5\\
		6 & -12 & 7
		\end{pmatrix} \ \ \p_2 = \begin{pmatrix}
		-3 & 6 & -3\\
		-5 & 10 & -5\\
		-6 & 12 & -6
		\end{pmatrix} = E - \p_1\\
		$
	\end{examples}
	\begin{defin}
		$(A_k) = ((a^k_{ij}))^\infty_{k=1}$ -- последовательность матриц\\
		$\exists \lim\limits_{k\rightarrow\infty}A_k = A = (a_{ij}) \Leftrightarrow \forall \ i, j \ \exists a_{ij} = \lim\limits_{k\rightarrow\infty} a^k_{ij}$\\
		$$S = \underbracket{\sum\limits_{m=1}^\infty A_m}_{\stackrel{\text{\textlarger[2]{Ряд.}}}{\text{Сумма ряда.}}} \overset{\exists}{=} \lim\limits_{N\rightarrow\infty}\underbracket{\sum\limits_{m=1}^N A_m}_{\stackrel{S_N \text{ \textlarger[2]{частичная}}}{\text{ сумма ряда}}}\\
		$$
		$$
		f(x)\text{ аналитическая в }|x|<R \Leftrightarrow f(x) = \sum_{m=0}^\infty C_m(x)^m \ \ \ C_m = \frac{f^{(m)}(0)}{m!}
		$$
	\end{defin}
	Ряд Тейлора.\\
	$\mathlarger{e^x = \sum\limits_{m=0}^\infty \frac{x^m}{m!} \ R = \infty \ \cos x = \sum\limits_{m=1}^\infty \frac{(-1)^m x^{2m}}{(2m)!} \ \ \ R = \infty}\\
	\mathlarger{
		\ln(1+x) = \sum\limits_{m=1}^\infty \frac{(-1)^{m-1} x^m}{m} \ \ |x| < 1 \ \ \ \text{либо } x = 1}$
	\begin{defin}
		Функция от матрицы.\\
		$A_{n\times n}\\
		f(A) = \sum\limits_{m=0}^\infty C_m A^m$, где $\boxed{\begin{array}{rcl}
			C_m & = & \frac{f^{(m)}(0)}{m!}\\
			f(x) & = & \sum\limits_{m=0}^\infty C_m x^m
			\end{array}}$
	\end{defin}
	$e^A = \sum\limits_{m=0}^\infty \frac{A^m}{m!}\\
	\cos A = \sum\limits_{m=0}^\infty \frac{(-1)^m}{(2m)!}A^{2m}$
	\begin{theorem}
		$f$ аналитическая в $|x| < R\\
		A_{n\times n} \ \ $ все с.ч. $|\lambda| < R$\\
		\textbf{$A$ диагонализируемая} То есть:\\
		$\exists \underset{\text{невырожд.}}{T}: \Lambda = T^{-1}A T\\
		\exists \p_\lambda: A = \sum\limits_{\lambda}\lambda \p_\lambda$\\
		$\Downarrow$
		\begin{mylist}
			\item 
			$\underset{f(A)}{\exists} = T\begin{pmatrix}
			f(\lambda_1) & \ldots & 0\\
			\vdots & \ddots & \vdots\\
			0 & \ldots & f(\lambda_n)
			\end{pmatrix} T^{-1}$
			\item 
			$\underset{f(A)}{\exists} = \sum\limits_{\lambda\text{с.ч.}}f(\lambda)\p_\lambda$
		\end{mylist}
	\end{theorem}	
	\begin{proof}
		\ \\
		\begin{mylist}
			\item 
			\belowbaseline[-10pt]{
				\begin{minipage}{0.3\textwidth}
					$f(A) = \sum\limits_{m=0}^\infty C_m A^m\\
					A^m = (T\Lambda T^{-1})^m = $
				\end{minipage}
				\begin{minipage}{0.4\textwidth}
					$\boxed{\begin{array}{l}f(x) = \sum\limits_{m=0}^\infty C_m x^m\\
						|x|<R
						\end{array}}$
			\end{minipage}}\\\\
			$= T\Lambda \underbrace{T^{-1} T}_{E} \Lambda T^{-1}\ldots T \Lambda T^-1 = \\
			= T\Lambda^m T^{-1} = T\begin{pmatrix}
			\lambda_1^m & 0\\
			0 && \lambda_n^m
			\end{pmatrix} T^{-1}\\
			f(A) = \sum\limits_{m=0}^\infty C_m T\Lambda^m T^{-1} = T(\sum\limits_{m=0}^\infty C_m\Lambda^m) T^{-1} = \\
			= T\begin{pmatrix}
			\sum\limits_{m=0}^\infty C_m \lambda^m_1 & \ldots & 0\\
			\vdots & \ddots & \vdots\\
			0 & \ldots & \sum\limits_{m=0}^\infty C_m\lambda^m_n
			\end{pmatrix} T^{-1} =
			T\begin{pmatrix}
			f(\lambda_1) & 0\\
			0 & f(\lambda_n)
			\end{pmatrix} T^{-1}\\
			|\lambda_i| < R
			$
			\item 
			$A^m = (\sum\limits_{\lambda\text{с.ч.}}\lambda\p_\lambda)^m \underset{\stackrel{\mathlarger{\lambda\neq \mu}}{\p_\lambda \p_\mu - \0}}{=} \sum\limits_{\lambda}\lambda^m\p^m_\lambda = \sum\limits_{\lambda}\lambda^m \p_\lambda\\
			f(A) = \sum\limits_{m=0}^\infty C_m(\sum\limits_\lambda \lambda^m \p_\lambda
			) = \sum\limits_\lambda (\underset{|\lambda| < R}{\sum\limits_{m=0}^{\infty}C_m\lambda^m = f(\lambda)})\p_\lambda = \sum\limits_\lambda f(\lambda) \p_\lambda$
		\end{mylist}
	\end{proof}
	\begin{remark}
		$A$ диагон. $\Leftrightarrow A = T\Lambda T^{-1}\\
		\Leftrightarrow A = \sum\limits_{\lambda\text{с.ч.}}\lambda\p_\lambda\\
		f(x) = \sum\limits_{m=0}^\infty c_m x^m\\
		f(A) = T\begin{pmatrix}
		f(\lambda_1) & 0\\
		0 & f(\lambda_n)
		\end{pmatrix} T^{-1}\\
		f(A) = \sum\limits_{\lambda\text{с.ч.}}f(\lambda)\p_\lambda\\
		t\in \R\\
		f(At) = \sum\limits_{m=0}^\infty C_m A^m t^m\\
		t^m A^m = t^m T\Lambda^m T^{-1} = T\begin{pmatrix}
		(\lambda_1 t)^m & 0\\
		0 & f(\lambda_n t)
		\end{pmatrix} T^{-1}\\
		\boxed{
			f(At) = T\begin{pmatrix}
			f(\lambda_1 t) & 0\\
			0 & f(\lambda_n t)
			\end{pmatrix}T^{-1}
		}\\\\
		t^m A^m = \sum\limits_{\lambda\text{с.ч.}}t^m\lambda^m \p_\lambda\\
		\boxed{f(At) = \sum\limits_{\lambda\text{с.ч.}}f(t\lambda)\p_\lambda}$
	\end{remark}
	\begin{examples}
		$e^{At}\\
		A = \begin{pmatrix}
		7 & -12 & 6\\
		10 & -19 & 10\\
		12 & -24 & 13
		\end{pmatrix}\\
		\chi(t) = det(A-tE) = (t-1)^2(t+1)\\
		\lambda_1 = 1 \ \ \alpha(\lambda_1) = 2\\
		\lambda_2 = -1 \ \ \alpha(\lambda_2) = 1\\
		V_{\lambda_1}: \begin{pmatrix}
		6 & -12 & 6 & \vline \ 0\\
		10 & -20 & 10 & \vline \ 0\\
		12 & -24 & 12 & \vline \ 0
		\end{pmatrix}
		\\
		V_{\lambda_1} = span \underset{v_1}{\begin{pmatrix}
			2\\1\\0
			\end{pmatrix}} \underset{v_2}{\begin{pmatrix}
			1\\0\\-1
			\end{pmatrix}} \ \ \gamma(\lambda_1) = 2\\
		V_{\lambda_2}: \left(\begin{array}{ccc|c}
		8 & -12 & 6 & 0\\
		10 & -18 & 10 & 0\\
		12 & -24 & 14 & 0
		\end{array}\right)\\
		V_{\lambda_2} = span\underset{v_3}{\begin{pmatrix}
			3\\5\\6
			\end{pmatrix}} \ \ \gamma(\lambda_2) = 1\\
		\forall \ \lambda: \left.\begin{array}{rcl}
		\alpha(\lambda) & = & \gamma(\lambda)\\
		\sum\limits_\lambda \alpha(\lambda) & = & 3	
		\end{array}\right\} \Rightarrow A \text{ диагонализируемая}\\
		T_{e\rightarrow v} = (v_1 v_2 v_3) = \begin{pmatrix}
		2 & 1 & 3\\
		1 & 0 & 5\\
		0 & -1 & 6
		\end{pmatrix}\\
		A = T\begin{pmatrix}
		1 & 0 & 0\\
		0 & 1 & 0\\
		0 & 0 & 1
		\end{pmatrix} T^{-1}\\
		e^{At} = T\begin{pmatrix}
		e^t & 0 & 0\\
		0 & e^t & 0\\
		0 & 0 & e^t
		\end{pmatrix} T^{-1}
		= \begin{pmatrix}
		4e^t - 3e^{-t} & -6e^t + 6e^{-t} & 3e^t-3e^{-t}\\
		5e^t - 5e^{-t} & -9e^t + 10e^{-t} & 5e^t - 5e^{-t}\\
		6e^t - 6e^{-t} & -12 e^t + 12e^{-t} & 7e^t - 6e^{-t}
		\end{pmatrix}\\
		\p_i : V\underset{i = 1, 2}{\rightarrow} V_{\lambda_i} \subset V\\
		\p_1 = T\left(\begin{array}{cc|c}
		1 & 0 &0\\
		0 & 1 &0\\
		\cline{1-2}
		0 &\multicolumn{1}{c}{0} & 0
		\end{array}\right)T^{-1} = \begin{pmatrix}
		4 & -6 & 3\\
		5 & -9 & 5\\
		6 & -12 & 7
		\end{pmatrix} \ \ Im\p_1 = span(v_1, v_2) = V_{\lambda_1}\\
		\p_2 = T\left(\begin{array}{cc|c}
		0 & \multicolumn{1}{c}{0} & 0\\
		0 & \multicolumn{1}{c}{0} & 0\\
		\cline{3-3}
		0 & 0 & 1
		\end{array}\right) T^{-1} = \begin{pmatrix}
		-3 & 6 & -3\\
		-5 & 10 & -5\\
		-6 & 12 & -6
		\end{pmatrix} \ \ Im\p_2 = span(v_3) = V_{\lambda_2}\\
		A = 1\cdot\begin{pmatrix}
		4 & -6 & 3\\
		5 & -9 & 5\\
		6 & -12 & 7
		\end{pmatrix} + (-1)\cdot\begin{pmatrix}
		-3 & 6 & -3\\
		-5 & 10 & -5\\
		-6 & 12 & -6
		\end{pmatrix}\\
		e^{At} = e^t\cdot\begin{pmatrix}
		4 & -6 & 3\\
		5 & -9 & 5\\
		6 & -12 & 7
		\end{pmatrix} + e^{-t}\cdot\begin{pmatrix}
		-3 & 6 & -3\\
		-5 & 10 & -5\\
		-6 & 12 & -6
		\end{pmatrix}\\
		A_{n\times n} \ \ \ \ x = \begin{pmatrix}
		x_1(t)\\\vdots\\ x_n(t)
		\end{pmatrix} \ \ \ \dot x \text{ -- производная}\\
		\dot{x} = \begin{pmatrix}
		\dot x_1 (t)\\
		\dot x_2(t)\\
		\vdots\\
		\dot x_n(t)
		\end{pmatrix}\\
		\boxed{\dot x = Ax} \ \ \ \; \ \ \ x = e^{At}C \ \ \ C = \begin{pmatrix}
		c_1\\
		\vdots\\
		c_n
		\end{pmatrix}\\
		\text{с.л.д.у. с постоянным коэффициентом однородности} \ \ \begin{array}{c}(e^{At})' = Ae^{At}\\e^{A\cdot 0} = E\end{array}\\
		e^{At} = \left(\sum\limits_{\lambda\text{с.ч.}} e^{\lambda t} \p_\lambda\right)' = \underline{\underline{\sum\limits_{\lambda\text{ с.ч.}} \lambda e^{\lambda t} \p_\lambda}}\\\\
		A\cdot e^{At} = \sum\limits_{\mu} \mu \p_\mu \cdot \sum\limits_\lambda e^{\lambda t} \p_\lambda \underset{\mu = \lambda}{=}\underline{\underline{\sum\limits_\lambda \lambda e^{\lambda t}\p_\lambda}}$
	\end{examples}\newpage
	\begin{remark}
		$\exists \ A^{-1} \Leftrightarrow detA \neq 0 \Leftrightarrow$ \belowbaseline[-12pt]{$\begin{array}{l}
			\text{все с.ч. } \lambda\neq 0\\
			\text{(все корни хар. многочлена)}
			\end{array}$
		}
	\end{remark}
	$\pu A$ диагонализируема. Все с.ч. $\lambda\neq 0\\
	A^{-1} = T\Lambda^{-1} T^{-1} = T\begin{pmatrix}
	\frac{1}{\lambda_1} & 0\\
	0 & \frac{1}{lambda_n}
	\end{pmatrix}T^{-1}\\
	\Lambda\Lambda^{-1} = E\\
	AA^{-1} = \underbracket{T\underbracket{\Lambda \underbracket{T^{-1}T}_E\Lambda^{-1}}_E T^{-1}}_E = E\\
	A^{-1} = \sum\limits_{\lambda\text{с.ч.}}\frac{1}{\lambda}\p_\lambda\\
	(AA^{-1} = E \text{ \underline{упр.}})\\
	\sqrt[m]{A} = T\sqrt[m]{\Lambda} T^{-1} = T\begin{pmatrix}
	\sqrt[n]{\lambda_1} & \ldots & 0\\
	\vdots & \ddots & \vdots\\
	0 & \ldots & \sqrt[m]{\lambda_n}
	\end{pmatrix} T^{-1}\\
	\pu \text{все }\lambda_i \geq 0\\
	(m\text{ нечет }\Rightarrow \lambda \text{ любого знака})\\
	(\sqrt[m]{\Lambda})^m = \Lambda\\
	(\sqrt[m]{A})^m = T\underbracket{ \sqrt[m]{\Lambda} \underbracket{T^{-1}T}_E \sqrt[m]{\Lambda} 
		\underbracket{T^{-1} \ldots T}_E \sqrt[m]{\Lambda} }_\Lambda T^{-1} = T\Lambda T^{-1} = A\\
	\boxed{\sqrt[m]{A} = \sum\limits_{\lambda\text{с.ч.}}\sqrt[m]{\lambda} \p_\lambda}\\
	(\text{упр.: }(\sqrt[m]{A})^m = A)
	$
	\begin{examples}
		$A = \begin{pmatrix}
		7 & -12 & 6\\
		10 & -19 & 10\\
		12 & -24 & 13
		\end{pmatrix}\\
		\begin{matrix}\lambda_1 = 1\\\lambda_2 = -1\end{matrix} \ \ A^{-1} = T\begin{pmatrix}
		1 & 0 & 0\\
		0 & 1 & 0\\
		0 & 0 & -1
		\end{pmatrix}T^{-1}\\
		A^{-1} = \mathlarger{\frac{1}{1}}\p_1 + \mathlarger{\frac{1}{(-1)}}\p_2 = \p_1 -\p_2 = A\\
		A^2 = E$
	\end{examples}
\end{document}