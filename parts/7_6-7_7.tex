\documentclass[../main.tex]{subfiles}
\begin{document}
	\subsection{Комплексификаци линейного вещ. пространства. Продолжение вещ. линейного оператора.}
	$\A \in End(V) \ \ \ V $ над полем $K$\\
	$\begin{array}{lll}
	\multicolumn{3}{c}{\chi(\lambda) \underset{\lambda \text{ корень}}{= 0}}\\
	\multicolumn{2}{c}{\swarrow} & \searrow \text{III}\\
	\multicolumn{2}{c}{
		\begin{array}{l}
		K = \R/\mathbb{C}\\
		\text{Все корни }\lambda\in K\\
		\text{Т.е. каждый корень с.ч. }\\
		\sum\limits_{\lambda\text{с.ч}} \alpha(\lambda) = n = dim V
		\end{array}
	}
	& \begin{array}{l}
	K = \R\\
	\text{Не все корни вещ.}\\
	\text{т.е. }\exists \lambda \not\in K = \R\\
	\sum\limits_{\text{вещ.} \lambda \text{с.ч.}}\alpha(\lambda) < n = dim V\\
	\A \rightarrow A ?
	\end{array}\\
	\multicolumn{2}{c}{
		\begin{array}{rl}
		\begin{array}{r}
		\text{I}\swarrow\\
		\forall \ \lambda: \gamma(\lambda) = \alpha(\lambda)\\
		\A \text{ -- о.п.с.} \rightarrow A \text{ диагонализир.}
		\end{array} & 
		\begin{array}{l}\\
		\searrow \text{II}\\
		\exists \ \lambda: \gamma (\lambda) < \alpha(\lambda)\\
		\A \text{ не о.п.с.} \\
		\rightarrow A \text{ приводится к Жордановой форме}
		\end{array}
		\end{array}
	}
	\end{array}$
	\begin{defin}
		$V$ -- линейное пространство над $\R\\
		\forall \ x, y \in V \ \ \ v:= x + iy \ \in V_\mathbb{C}\\
		\forall \ v, v' \in V_\mathbb{C}: \ \ \ \ \ \ \ \begin{matrix}
		x = Re \ v\\
		y = Im \ v
		\end{matrix}\\
		\text{Определим}$
		\begin{mylist}
			\item $v = v' \Leftrightarrow \left[\begin{matrix}
			x = x' \in V\\
			y = y'
			\end{matrix}\right.$
			\item 
			$v + v' = \omega = a + bi \in V_\mathbb{C}\Leftrightarrow \left[ \begin{matrix}
			a = x + x' \in V\\
			b = y + y'
			\end{matrix}\right.$
			\item 
			$\forall\ \lambda = \alpha + i\beta \in \mathbb{C}, \ \ \ \alpha, \beta \in \R\\
			a + bi = \omega = \lambda\cdot v \Leftrightarrow(\alpha + i\beta)(x + iy) = \underbrace{\overbrace{\alpha x - \beta y}^{\mathlarger{a \in V}} + i\overbrace{\beta x + \alpha y}^{\mathlarger{b \in V}}}_{\mathlarger{\in V_\mathbb{C}}}
			$
			\item 
			$\forall \ x\in V \leftrightarrow x + i\0 \in V_{\mathbb{C}}\\
			V \subset V_\mathbb{C}\\
			\0 \leftrightarrow \0 + i\0$
		\end{mylist}
		\underline{Упр.:} $V_\mathbb{C}$ -- линейное пространство над $\mathbb{C}$\\
		$\boxed{V_\mathbb{C} \text{ -- комплексификация линейного вещественного пространства } V}$
	\end{defin}
	\begin{stat}
		$e_1\ldots e_n$ базис $V \Rightarrow e_1 \ldots e_n$ базис $V_\mathbb{C}$\\
		Т.е. $dim V = dim V_\mathbb{C} = n\\
		V \subset V_\mathbb{C}\ \ \ \ \ $ структуры над \underline{разными} полями.
	\end{stat}
	\begin{proof}
		$e_1\ldots e_n$ базис $V_\mathbb{C}?$\\
		-- порождающая?\\
		-- линейно независимая?\\
		\begin{mylist}
			\item 
			$\forall \ v\in V_\mathbb{C} \ \ v = x\in V + iy\in V = \sum\limits_{j=1}^n x_j e_j + i\sum\limits_{j=1}^n y_j e_j = \\
			\sum\limits_{j=1}^n \underset{\alpha_j \in \mathbb{C}}{\boxed{x_j + iy_j}}e_j \Rightarrow e_1\ldots e_n $ порождающая.
			\item
			$\sum\limits_{j=1}^n \gamma_j e_j = \0 \ \ \ \begin{matrix}
			\gamma_j \in \mathbb{C}\\
			\gamma_j = \alpha_j + i\beta_j
			\end{matrix}\\
			||\\
			\underbracket{\sum\limits_{j=1}^n\overset{\text{вещ.}}{\alpha_j}e_j}_x +
			i \underbrace{\sum\limits_{j=1}^n \overset{\text{вещ.}}{\beta_j}e_j}_y = \0\\
			\Leftrightarrow \left\{
			\begin{array}{lcc}
			x = \0 = & \sum\limits_{j=1}^n\overset{\text{вещ.}}{\alpha_j}e_j & \Leftrightarrow\\
			y = \0 = & \sum\limits_{j=1}^n \overset{\text{вещ.}}{\beta_j}e_j &\overset{\mathlarger{e_1\ldots e_n\text{ линейно независ.}}}{\Leftrightarrow}
			\end{array}
			\right.
			\left\{\begin{array}{c}
			\forall \ j \ \alpha_j = 0\\
			\forall \ j \ \beta_j = 0
			\end{array}\right. 
			\Leftrightarrow \forall j\ \gamma_j = 0\\\\
			\Rightarrow \underset{\mathlarger{\text{лин. незав.}}}{e_1\ldots e_n}$ в $V_\mathbb{C}$
		\end{mylist}
	\end{proof}
	\begin{defin}
		$z = x + iy \ \ x, y\in V$\\
		\begin{minipage}{0.5\textwidth}
			\textbf{вектор сопряженный к $z$:}\\
			$\vec{z} = x-iy$\\
			$(\vec{\vec z} = z, (\vec{z_1 + z_2}) = \vec z_1 + \vec z_2, \vec{(\lambda z)} = \vec \lambda \vec z)$\\
			\vfill
		\end{minipage}
		\begin{minipage}{0.5\textwidth}
			$z = \begin{pmatrix}
			z_1\\\vdots\\z_n
			\end{pmatrix}\\
			\vec{z} = \begin{pmatrix}
			\vec z_1\\\vec z_2 \\ \vdots \\ \vec z_n
			\end{pmatrix}$
		\end{minipage}\\
	\end{defin}
	\begin{stat}
		$v_1\ldots v_m $ линейно незав. в $V_\mathbb C \Rightarrow \vec v_1\ldots \vec v_m$ линейно независимы в $V_\mathbb C$\\
		Очевидно, $v_1\ldots v_m$ линейно зависимы $\Rightarrow \vec v_1 \ldots \vec v_m$ линейно зависимы.
	\end{stat}
	\begin{proof}\ \\
		$\left.\begin{array}{ll}
		\vec{\sum\limits_{j=1}^m \gamma_j \vec v_j} & = \vec \0 = \0\\
		||\\
		\sum\limits_{j=1}^m \vec \gamma_j \vec{\vec v_j}& = \sum\limits_{j=1}^m \underset{\text{линейно незав.}}{\gamma'_j v_j}
		\end{array}\right| \Leftrightarrow \forall j \ \gamma'_j = 0 = \vec \gamma_j \Leftrightarrow \gamma_j = 0 \\\\
		\Rightarrow$ линейно независим.
	\end{proof}
	$\boxed{rg(v_1\ldots v_m) = rg(\vec v_1 \ldots \vec v_m)}$
	\begin{defin}
		$\A \in End(V)\\
		V_\mathbb{C}\\
		\forall v = x\in V + i\underset{\in V}{y} \in V_\mathbb{C} \ \ \ \A_\mathbb{C} v = \A x \in V + i\underset{\in V }{\A y} \in V_\mathbb{C}\\
		\A_\mathbb{C}: V_\mathbb{C}\rightarrow V_\mathbb{C}\\
		\A_\mathbb{C} \in End(V_\mathbb{C})$\\
		Линейность?
		\begin{mylist}
			\item 
			Аддитивность. $\A_\mathbb{C}(v_1 + v_2) = \A_\mathbb{C} v_1 + \A_\mathbb{C} v_2$\\
			Очевидно, из аддитивности $\A\\
			v_1 + v_2 = (x_1 + x_2) + i(y_1 + y_2)$
			\item Однородность\\
			$\forall \lambda = \alpha + i\beta \in \mathbb{C} \ \ \ \alpha, \beta \in \R\\
			\A_\mathbb{C}(\lambda v) = \A_\mathbb{C}((\alpha + i\beta)(x + iy)) = \\
			= \A_\mathbb{C}((\alpha x - \beta y) + i(\alpha y + \beta x))= \\
			= \A (\alpha x - \beta y) + i \A(\alpha y + \beta x) = \\
			= \alpha \A x - \beta \A y + i \alpha \A y + i \beta \A x = \\
			= (\alpha + i \beta) \A x + i(\alpha + i\beta) \A y = \lambda \A x + i \lambda \A y =\\
			= \lambda(\A x + i \A y) = \lambda \A_\mathbb{C}v$
		\end{mylist}
		$\A_\mathbb{C}$ -- \textbf{продолжение линейного вещ. оператора $\A$}\\
		с пространства $V$ на его комплексификацию $V_\mathbb{C}$
	\end{defin}
	\textbf{Свойства $\A_\mathbb{C}$:}
	\begin{mylist}
		\item 
		\belowbaseline[-12pt]{
			$\left.
			\begin{array}{c}
			\underset{\text{веществ.}}{e_1\ldots e_n}$ базис $V(V_\mathbb{C})\\
			\A \leftrightarrow A\\
			\A_\mathbb{C} \leftrightarrow A_\mathbb{C}
			\end{array}
			\right\} \Rightarrow A_\mathbb{C} = A$}\\
		Т.е. $\A_\mathbb{C}$ в вещ. базисе имеет вещ. матрицу, совпадающую с матр. $\A$
		\item $\forall z \in V_\mathbb{C} \ \ \vec{\A_\mathbb{C} z} = \A_\mathbb{C}\vec z\\
		z = x + iy \ \ \vec{\A_\mathbb{C} z} = \vec{\underset{\text{вещ.}}{\A x} + i \underset{\text{вещ.}}{\A y}} = \A x - i \A y = \\
		= \A x + i \A (-y) = \A_\mathbb{C}(x-iy) = \A_\mathbb{C} \vec z$
		\item
		\belowbaseline[-12pt]{
			$ 
			\begin{array}{cccc}
			\chi_\A(t) & =&  \chi_{\A_\mathbb{C}}(t) & \pu e_1\ldots e_n \text{базис} V\\
			|| & & || & \A \leftrightarrow A\\
			det(A-tE) & & det(A_\mathbb{C}-tE) & \A_\mathbb{C} \leftrightarrow A_\mathbb{C} = A
			\end{array}$}\\
		Все корни характеристического многочлена $\chi_\A$ являются собственными числами $\A_\mathbb{C}$
		\item 
		$\chi_\A(\lambda) = \chi_{\A_\mathbb{C}}(\lambda) = 0$\\
		Т.к. многочлен с вещ. коэф. $\Rightarrow \vec \lambda$ тоже корень.\\
		$\lambda = \alpha + i\beta \ \ \text{корень } \chi_{\A_\mathbb{C}} \ \ \ \ \chi_{\A_\mathbb{C}}(\vec \lambda) = 0\\
		v \text{ соотв. с.в.}\\
		\Rightarrow \vec v \text { с.в. для }\vec \lambda = \alpha - i\beta\\
		\boxed{\text{для }\A_{\mathbb{C}}: 
			\begin{matrix}dim V_\lambda = dim V_{\vec \lambda} (\text{из утв. 2})\\
			\gamma(\lambda) = \gamma(\vec \lambda)
			\end{matrix}}\\\\
		\A_\mathbb{C} \vec v \underset{\text{св-во 2}}{=} \vec{\A_\mathbb{C} \underset{\stackrel{\uparrow}{\text{с.в. для }\lambda}}{v}} = \vec{\lambda v} = \vec \lambda \vec v \Rightarrow \vec v \text{ с.в. для } \vec \lambda$
	\end{mylist}\ \\
	\textbf{"III": }$\A\in End(V)\\
	V$ над $\R$\\
	$\sum\limits_{\lambda\text{с.ч.}}\alpha(\lambda) < n = dim V$\\
	Т.е. не все корни $\chi_\A$ \textbf{вещ.}\\
	$\rightarrow \text{строим} \A_\mathbb{C}\in End(V_\mathbb{C}) \ \ \ A_\mathbb{C} = A$\\
	Все корни с.ч. $\Rightarrow$ матрица для $A_\mathbb{C}$ будет сведена либо к I, либо к II\\
	\begin{examples}
		\belowbaseline[-12pt]{$A = \begin{pmatrix}
			4 & -5 & 7\\
			1 & -4 & 9\\
			-4 & 0 & 5
			\end{pmatrix}$}\\
		$\chi_A(t) = det(A-tE) = -(t-1)(t^2-4t + 13)\\
		D = -36 < 0\\
		\lambda_1 = 1 \text{ с.ч. }\alpha(\lambda_1) = 1 \ \ \ \ \ \lambda_{2, 3} = 2+\pm i3 \ \alpha(2, 3) = 1\\
		A_\mathbb{C} = A: \lambda_{2, 3} = 2\pm i\\
		\lambda_1 = 1 \ \ V_{\lambda_1} ] span\begin{pmatrix}
		1\\2\\1
		\end{pmatrix}\\
		\lambda_2 = 2+3i \ \ \ \ \ \; 1\leq \gamma(\lambda_2) \leq \alpha(\lambda_2) = 1 \Rightarrow \gamma(\lambda_2) = 1$\\
		Решаем СЛОУ методом Гаусса точно так же, как мы решали для вещ. чисел.\\ Только теперь арифметические операции с комплексными.\\
		$V_{\lambda_2} = span\begin{pmatrix}
		3-3i\\
		5-3i\\
		4
		\end{pmatrix}\\
		\lambda_3 = 2-3i \ \ \; \; V_{\lambda_3} = span\begin{pmatrix}
		3+3i\\
		5+3i\\
		4
		\end{pmatrix} = v_3\\
		\forall \lambda: \ \gamma(\lambda) = \alpha(\lambda) \Rightarrow A_\mathbb{C} = A \text{ диагонализир.}\\
		T_{e\rightarrow v} = \begin{pmatrix}
		1 & 3-3i & 3+3i\\
		2 & 5-3i & 5+3i\\
		1 & 4 & 4
		\end{pmatrix}\\
		T^{-1} A T = \begin{pmatrix}
		1 & 0 & 0\\
		0 & 2+3i & 0\\
		0 & 0 & 2-3i
		\end{pmatrix}T^{-1} = \ldots
		$
	\end{examples}
	\subsection{Минимальный многочлен. Теорема Кэли-Гамильтона}
	\begin{defin}
		Нормализованный (старший коэф. = 1) многочлен $\psi(t)$ называется \\\textbf{аннулятором элемента $v\in V$}, если $\psi(\A) v = \0$
	\end{defin}
	$\psi(t) = t^m + a_{m-1}t^{m-1} + \ldots + a_1 t + a_0\\
	\psi(\A) = \A^t + a_{m-1}\A^{m-1} + \ldots + a_1 \A + a_0 \E \in End(V)\\
	\A^0 = \E\\
	\psi(t) = \prod\limits_{\lambda \text{ корень}}(t-\lambda)^{m(\lambda)}\\
	(\A-\lambda\E)^{m(\lambda)} \cdot (\A - \mu\E)^{m(\mu)} = (\A-\mu\E)^{m(\mu)} \cdot (\A -\lambda\E)^{m(\lambda)}\\
	\A^k\E^r = \E^r\A^k$\\
	Т.е. перестановочны.
	\begin{defin}
		$\psi(t)$ аннулятор элемента $v\in V $ наименьшший возможной степени \\называется \textbf{минимальным аннулятором элемента $v$}
	\end{defin}
	\begin{theorem}[О минимальном аннуляторе элемента]\ \\
		$\A\in End(V)$
		\begin{mylist}
			\item $\forall v \in V\  \exists! $ минимальный аннулятор $v$
			\item $\forall$ аннулятор элемента делится на его минимальный.
		\end{mylist}
	\end{theorem}
	\begin{proof}\ \\
		\begin{mylist}
			\item 
			\begin{mylist}
				\item 
				$\pu v = \0 \ \ \ \; \psi(t) = 1 \ \ \ \ \ \; $ Очевидно, минимальный аннулятор.\\
				$\psi(\A) v = \E v = \0$
				\item 
				$\pu v \neq \0\\
				\underbracket{\underbracket{(\E)v, \A v, \A^2 v, \ldots, \A^{m-1} v,}_{\text{\normalsize{линейно независимая система}}} \A^m v}_{\text{\normalsize{линейно зависимая система}}}$\\\\
				$dim V = n\\
				m\leq n+1\\
				\A^m v = \sum\limits_{k=0}^{m-1} a_k \A^k v\\
				\0 = \A^m v - \sum\limits_{k=0}^{m-1} a_k \A^k v = (\A^m - \sum\limits_{k-0}^{m-1}a_k\A^k)v \leftarrow \text{ Алгоритм}\\
				\psi(t) = t^m - \sum\limits_{k=0}^{m-1} a_k t^k$\\
				Очевидно, по построению это минимальный аннулятор элемента $v$
			\end{mylist}
			\item 
			$\psi_1$ -- аннулятор $v\\
			\psi_1(t) = a(t) \psi(t) + r(t)\\
			deg \ r(t) < deg \ \psi (t)\\
			\0 = \psi_1(\A) v = (a(\A)\psi(\A) + r(\A))v
			= a(\A) \underbracket{\psi(\A)v}_{=\0} + r(\A) v = r(\A) v\Rightarrow \\
			\Rightarrow \left.\begin{matrix}
			r(t) \text{ аннулятор } v\\deg \ r < deg \ \psi
			\end{matrix}\right\} \Rightarrow \text{Противоречие с минимальностью }\psi \Rightarrow\\
			\Rightarrow r(t) \equiv 0 \Rightarrow \psi_1 \vdots \psi$
		\end{mylist}
	\end{proof}
	\begin{defin}
		Нормализованный многочлен $\phi(t)$ \textbf{называется аннулятором } $\A$,\\ если $\phi(\A) = 0\\
		(\Leftrightarrow \forall v \in V \ \ \phi(\A) v = \0)$\\
		Аннулятор $\A$ минимальной степени называется \textbf{минимальный многочленом}
	\end{defin}
	
	\begin{theorem}[о минимальном многочлене]
		$\A \in End(V)$
		\begin{mylist}
			\item $\forall \A \ \exists ! $ минимальный многочлен
			\item $\forall$ аннулятор $\A$ делится на минимальный многочлен
		\end{mylist}
	\end{theorem}
	\begin{proof}\ \\
		$e_1\ldots e_n$ базис $V\\
		\Rightarrow $ по Теореме 1 для $\forall e_j \ \exists! \ \psi_j $ минимальный аннулятор $e_j\\
		\psi_j(\A) e_j = \0\\
		\psi(t) = $ Н.О.К. ($\psi_1\ldots\psi_n$)\\
		$\forall v \in V \ \ \phi(\A) v = \phi(\A) \sum\limits_{i=1}^n v_i e_i = \sum\limits_{i=1}^n v_i \phi(\A) e_i = \\
		= \sum\limits_{i=1}^n v_i \xi_i(\A) \underbracket{\psi_i (\A) e_i}_{=\0} = \0\\
		\phi\vdots \psi_j \Leftrightarrow \phi(t) = \xi_j(t)\psi_j(t)\\\\
		\Rightarrow \phi(\A) = \0 \Rightarrow \phi$ аннулятор $\A$\\
		Давайте покажем, что у $\phi$ степень минимальная.\\
		От противного.\\
		$\exists \phi_1 $  аннулятор $\A \ \ \ \ \pu deg \ \phi_1 < deg \ \phi\\
		\forall e_j : \phi_1(\A) e_j = \0 \Rightarrow \phi_1 $ аннулятор элемента $e_j\overset{\text{по Теореме 1}}{\Rightarrow}\\
		\Rightarrow \underset{\text{аннулятор }e_j}{\phi_1}\vdots \underset{\text{минимальный аннулятор }e_j}{\psi_j} \Rightarrow \phi_1 \vdots \phi \Rightarrow deg \ \phi_1 \geq deg \ \phi$. Противоречие $\Rightarrow\\
		\Rightarrow deg \ \phi$ минимальный $\Rightarrow $ п.2 доказан, т.к. $\forall$ аннулятор $\A\vdots\phi$\\
		\textbf{Единственность?}\\
		$\pu \underset{\begin{matrix}
			\nwarrow \nearrow\\
			\text{нормализов.} \Rightarrow \text{ ст. коэф. 1}
			\end{matrix}}{\phi_1, \phi}$ минимальные аннуляторы одной степени.\\
		$deg(\phi_1 - \phi) < deg(\phi) = deg(\phi_1)\\
		\forall v \in V \ \ \ (\phi_1 - \phi)(\A) v = \underset{=\0}{\phi_1(\A)}v - \underset{=\0}{\phi(\A)}v = \0 \Rightarrow \\
		\Rightarrow \phi_1 - \phi$ аннулятор $\A$ меньшей степени $\Rightarrow$ противоречие \underline{минимальн.} 
	\end{proof}
	\begin{examples}
		$A = \begin{pmatrix}
		0 & 1 & 0\\
		-4 & 4 & 0\\
		-2 & 1 & 2
		\end{pmatrix}\ \ \ \; \ \phi = ? \text{  минимальный многочлен}\\
		e_1  = \begin{pmatrix}
		1\\0\\0
		\end{pmatrix} \ \ \; \ \phi_1 ?\\
		e_1 = \underbracket{\underbracket{\begin{pmatrix}
				1\\0\\0
				\end{pmatrix} \ \ \; \A e_1 = \begin{pmatrix}
				0\\-4\\2
				\end{pmatrix}}_{\text{\normalsize{линейно независ.}}} \ \ \; \ \A^2e_1 = \begin{pmatrix}
			-4\\-16\\-8
			\end{pmatrix}}_{\text{\normalsize{линейно завис.}}}\\
		\A^2 e_1 = -4e_1 + 4\A e_1\\
		\psi_1(t) = t^2 - 4t + 4 = (t-2)^2\\
		e_2 = \underbracket{\underbracket{\begin{pmatrix}
				0\\1\\0
				\end{pmatrix} \ \ \; \A e_2 = \begin{pmatrix}
				1\\4\\1
				\end{pmatrix}}_{\text{\normalsize{линейно независ.}}} \ \ \; \ \A^2e_2 = \begin{pmatrix}
			4\\12\\-4
			\end{pmatrix}}_{\text{\normalsize{линейно завис.}}}\\
		\A^2 e_2 = 4\A e_2 - 4 e_2\\
		\psi_2(t) = t^2 - 4t + 4 = (t-2)^2\\
		\underbracket{e_2 = \underbracket{\begin{pmatrix}
				0\\0\\1
				\end{pmatrix}}_{\text{\normalsize{лин. нез.}}} \ \ \; \A e_3 = \begin{pmatrix}
			0\\0\\2
			\end{pmatrix}}_{\text{\normalsize{линейно завис.}}}\\
		\A e_3 = 2e_3\\
		\psi_3(t) = t-2\\
		\phi(t) = \text{Н.О.К. }((t-2)^2, (t-2)) = (t-2)^2
		$
	\end{examples}
	\begin{theorem}[Кэли-Гамильтона]
		$\A \in End(V)\\
		\underset{\text{характерист. многочлен}}{\chi(t)} = det(\A-t\E) $ -- аннулятор $\A$
	\end{theorem}
	\begin{proof}
		$\chi(\A) = det(\A - \A) = \0$
	\end{proof}
	Я так и не понял это норм доказательство или нет. В любом случае далее идет длинное док-во.
	\begin{proof}
		$\mu$ -- не корень $\chi(t)\\
		det (\A - \mu \E) \neq 0\\
		\Leftrightarrow \exists (\A -\mu \E)^{-1}\\
		e_1\ldots e_n$ базис $v$. $\A \leftrightarrow A\\
		(A-\mu E)^{-1} = \mathlarger{\frac{1}{det(A-\mu E)}}B \leftarrow$ союзная матрица (прис-ная)\\
		$B = (b_{ij}) \ \ \; \ b_{ij} = (-1)^{ij}M_{ij} \leftarrow $ определиитель $(n-1)$-го порядкка $A-\mu E$\\
		Т.е. мн-н степени $n-1$ относительно $\mu$\\
		$B = B_{n-1} \mu^{n-1} + B_{n-2} \mu^{n-2} + \ldots + B_1 \mu + B_0 \\
		\begin{array}{lc}
		det(A-\mu E) \cdot E = & (A-\mu E) (B_{n-1} \mu^{n-1} + \ldots + B_1 \mu + B_0)\\
		|| & \\
		\chi(\mu)\cdot E & \\
		|| & \\
		\sum\limits_{k=0}^n \alpha_k\mu^k \cdot E &
		\end{array}\\
		\begin{array}{rl|l}
		\mu^0: & \alpha_0 E = AB_0 & A^0\\
		\mu^1: & \alpha_1 E = AB_1 - B_0 & A^1\\
		\mu^2: & \alpha_2 E = AB_2 - B_1 & A^2\\
		\ldots & &\\
		\mu^{n-1}: & \alpha_{n-1} E = AB_{n-1} - B_{n-2} & A^{n-1}\\
		\mu^n: & \alpha_n E = -B_{n-1} & A^n
		\end{array}\\
		\chi(\A) = \chi(A) = \sum\limits_{k=0}^n \alpha_k A^k = AB_0 + A^2 B_1 - AB_0 + A^3 B_2 - A^2 B_1 + \ldots + A^n B_{n-1}\\ - A^{n-1} B_{n-2} - A^n B_{n-1} = \0\\
		\chi$ -- аннулятор $\A$
	\end{proof}\newpage
	\begin{theorem}
		$\A\in End(V)$\\
		Множество корнеq характеристического многочлена $\A$ совпадает с \\множеством корней минимального многочлена $\A$ (без учета кратности)
	\end{theorem}
	\begin{proof}
		$\chi(t)$ -- характерист., $\phi(t)$ -- минимальный многочлен.\\
		$"\Leftarrow" \; \; \pu \phi(\lambda) = 0\ \Rightarrow $ т.к. $\chi$ аннулятор $\A$, то по Т-ме 2 $\chi \vdots \phi \Rightarrow \chi(\lambda) = 0\\
		"\Rightarrow" \; \; \pu \chi(\lambda) = 0$
		\begin{mylist}
			\item $\pu \lambda\in K \Rightarrow \lambda$ с.ч. $\A \ \ \ \; \; \exists v\neq \0: \A v = \lambda V \Rightarrow\\
			\Rightarrow (\A-\lambda\E) v = \0 \Rightarrow \psi(t) = (t-\lambda)$ минимальный аннулятор $v$\\
			Т.к. $\phi\vdots\psi \Rightarrow \lambda $ корень $\phi\\
			\phi(\lambda) = 0$
			\item $\lambda \not \in K$ т.е. III случай: $K = \R$\\
			$\exists$ комплексные корни характерист. многочлена.\\
			$V\rightarrow V_\mathbb{C} \ \ \ \; e_1\ldots e_n $ базис $V \rightarrow$ базис $V_\mathbb{C}\\
			\A \rightarrow \A_\mathbb{C} \ \ \ \; \A_\mathbb{C} e_j = \A e_j + i \A \0 = \A e_j\\
			e_j = e_j + i\0\\
			\Rightarrow \forall k \ \A^k_\mathbb{C} e_j = \A^k e_j$ \\
			$\Rightarrow$ Применим алгоритм построения минимального многочлена (Теоремы 1, 2). \\
			Получим, что минимальные многочлены $\A_\mathbb{C}$ и $\A$ совпадают. \\
			$\left.\begin{array}{c}
			\text{Т.е. }\phi \text{ мин. мн-н для } \A_\mathbb{C}\\
			\chi_{\A_\mathbb{C}} = \chi_\A
			\end{array}\right\} \Rightarrow \text{Применим случай а) для }\A_\mathbb{C}\\
			\Rightarrow \lambda \text{ с.ч.} \ \lambda \text{ корень } \phi$
		\end{mylist}
	\end{proof}
	\begin{examples}
		$A = \begin{pmatrix}
		0 & 1 & 0\\
		-4 & 4 & 0\\
		-2 & 1 & 2
		\end{pmatrix}\\
		\chi(t) = \left|\begin{array}{cc|c}
		-t & 1 & 0\\
		-4 & 4-t & 0\\
		\cline{1-2}
		-2 & \multicolumn{1}{c}{1} & 2-t
		\end{array}\right| = (2-t)(t^2 - 4t + 4) = -(t-2)^3$\\
		Корни $\chi: 2$\\
		Корни $\phi: 2$\\
		$\leadsto$ еще один способ найти с.ч. -- \textbf{найти корни многочлена.}
	\end{examples}
	\begin{corollary}\
		\begin{mylist}
			\item $\underset{\text{\normalsize{характер. (аннулятор)}}}{\psi}\vdots\underset{\text{\normalsize{минимальный (аннулятор мин.)}}}{\phi}$
			\item $deg \ \phi = n = dim V \Rightarrow (-1)^n \chi = \phi\\
			\boxed{\begin{array}{rcl}
				\chi(t) & = & \prod\limits_\lambda (t-\lambda)^{\alpha(\lambda)}\\
				\phi(t) & = & \prod\limits_\lambda(t-\lambda)^{m(\lambda)}
				\end{array} \ 1\leq m(\lambda) \leq \alpha(\lambda)}$
		\end{mylist}
	\end{corollary}
\end{document}