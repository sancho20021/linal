\documentclass[../main.tex]{subfiles}
\begin{document}
	\subsection{Операторное разложение единицы. Корневые подпространства.}
	\begin{minipage}{0.5\textwidth}
		$\phi(t) = \prod\limits_\lambda (t-\lambda)^{m(\lambda)}$
	\end{minipage}
	\begin{minipage}{0.5\textwidth}
		$\sum\limits_\lambda m(\lambda) = m\\
		deg \ \phi = m$
	\end{minipage}\\
	$P_{m-1}$ -- линейное пространство многочленов степени не выше $m-1\\
	dimP_{m-1} = m\\
	\phi(t) = \underset{\overset{\mathlarger{\nwarrow \nearrow}}{\text{вз. просты}}}{(t-\lambda)^{m(\lambda)}\phi_\lambda(t)} \ \ \; 
	\begin{matrix}
	\phi_\lambda(t) = \prod\limits_{\mu\neq \lambda} (t-\mu)^{m(\mu)}\\
	\phi_\lambda(\lambda) \neq 0\\
	\phi_\lambda(\mu) = 0\\
	\mu \neq \lambda
	\end{matrix}$
	\begin{defin}
		$I_\lambda = \{p\in P_{m-1} | p\vdots \phi_\lambda\}$\\
		\textbf{Главный идеал}, порожденный многочленом $\phi_\lambda = \\
		= \{f\in P_{m(\lambda)-1}|p = f_\lambda \phi_\lambda  \}\\
		I_\lambda$ -- линейное подпространство $P_{m-1}\\
		p_{1, 2}\vdots \phi_\lambda \Rightarrow (p_1 + \alpha p_2)\vdots\phi_\lambda$
	\end{defin}
	\begin{theorem}
		$P_{m-1} = \bigoplus\limits_\lambda I_\lambda$
	\end{theorem}
	\begin{proof}\
		\begin{mylist}
			\item Дизъюнктность.\\
			$\0 = \sum\limits_\lambda \underbrace{f_\lambda\phi_\lambda}_{\in I_\lambda}  = 
			f_\lambda\cdot \phi_\lambda + \underbrace{\sum\limits_{\mu\neq \lambda}f_\mu \underbrace{\phi_\mu}_{\vdots (t-\lambda)^{m(\lambda)}}}_{\vdots (t-\lambda)^{m(\lambda)}}\\
			\Rightarrow f_\lambda \cdot \underset{\text{вз. просты}}{\phi_\lambda \vdots (t-\lambda)^{m(\lambda)}} \Rightarrow \underset{\stackrel{\uparrow}{deg \ f_\lambda = m(\lambda)-1}}{f_\lambda} \vdots (t-\lambda)^{m(\lambda)} \Rightarrow f_\lambda \equiv 0 \\
			\Rightarrow \forall \lambda \ \ \ f_\lambda \equiv 0 \Rightarrow f_\lambda \phi_\lambda \equiv \0 \Rightarrow \text{Дизъюнктны}$
			\item
			\belowbaseline[-12pt]{
				$\begin{array}{lcl}
				dim P_{m-1} & = & m\\
				|| & & \\
				\sum\limits_\lambda dim I_\lambda & = & \sum\limits_\lambda m(\lambda) = m
				\end{array}$}\\\\
			$I_\lambda \subset P_{m-1}\\
			\\
			\Rightarrow P_{m-1} = \bigoplus\limits_\lambda I_\lambda$
		\end{mylist}
	\end{proof}
	\begin{corollary}
		$\forall p \in P_{m-1} \ \exists! \ p = \sum\limits_\lambda p_\lambda\\
		p_\lambda \in I_\lambda\\
		\boxed{1 = \sum\limits_\lambda p_\lambda\text{ -- полиномиальное разложение единицы}}$ 
	\end{corollary}
	\newpage
	\begin{remark} \
		\begin{mylist}
			\item 
			$\lambda \neq \mu\\
			\begin{array}{rclcl}
			p_\lambda & \cdot & p_\mu & \vdots & \phi\\
			|| & & || & &\\
			f_\lambda\phi_\lambda & & f_\mu\phi_\lambda & = & \eta \cdot \phi\\
			& & \uparrow & & \\
			& & (t-\lambda)^{m(\lambda)} & &
			\end{array}$
			\item $\forall \lambda \ m(\lambda) = 1$\\
			\textbf{Если.} Т. е. все корни $\phi$ взаимно простые.\\
			$f_\lambda = const \ \ \ (def \ f_\lambda = m(\lambda) - 1 = 0)$
		\end{mylist}
	\end{remark}
	\begin{theorem}[Лагранжа] \ \\
		$\forall \lambda: m(\lambda) = 1 \Rightarrow \\
		\forall p \in P_{m-1} \ \ p(t) = \sum\limits_\lambda \frac{p(\lambda)}{\phi'(\lambda)}\cdot \phi_\lambda(t)$
	\end{theorem}
	\begin{proof}\ \\
		$\text{корень }\phi \rightarrow \mu \neq \lambda \ \ \begin{matrix}
			\phi_\lambda(\mu) = 0\\
			\phi_\lambda(\lambda) \neq 0
		\end{matrix}\\
		p(t) \ \sum\limits_\lambda p_\lambda(t) = $\belowbaseline[-12pt]{$\begin{array}{rcl}
			\sum\limits_\mu
			& \underset{\uparrow}{\boxed{f_\mu}} & \cdot\; \phi_\mu(t)\\
			\multicolumn{3}{c}{\begin{matrix}
				const,\text{ т.к. }\\
				\text{корни взаимно}\\
				\text{просты}
			\end{matrix}}
		\end{array}$}\\
		$p(\lambda) = f_\lambda \cdot \phi_\lambda(\lambda)
		\Rightarrow \forall \lambda : f_\lambda = \frac{p(\lambda)}{\phi_\lambda(\lambda)}\\
		\phi(t) = \prod\limits_\mu (t-\mu)\\
		\phi'(t) = \sum\limits_\mu \underbracket{\prod\limits_{\lambda\neq \mu} (t-\lambda)}_{\phi_\mu(t)} = 
		\sum\limits_\mu \phi_\mu(t)\\
		\phi'(\lambda) = \sum\limits_\mu$ \belowbaseline[-12pt]{$\begin{array}{cl}
			\phi_\mu(\lambda) & = \phi_\lambda(\lambda)\\
			\mathsmaller{||} & 
			\\ \mathsmaller{0 \; \mu \neq \lambda} & 
		\end{array}$} $
		\Rightarrow f_\lambda = \mathlarger{\frac{p(\lambda)}{\phi'(\lambda)}}
		\Rightarrow p = \sum\limits_\lambda \mathlarger{\frac{p(\lambda)}{\phi'(\lambda)}} \phi_\lambda(t)$
	\end{proof}
	\begin{corollary}
		$\forall \lambda: m(\lambda) = 1\\
		1 = \sum\limits_\lambda p_\lambda \Rightarrow \boxed{t = \sum\limits_\lambda \lambda p_\lambda}$
	\end{corollary}
	\begin{proof}
		По теореме: $1 = \sum\limits_\lambda \boldsymbol{p_\lambda} = \sum\limits_\lambda f_\lambda \cdot \phi_\lambda = \sum\limits_\lambda \mathlarger{\frac{1}{\boldsymbol{\phi'(\lambda)}}} \cdot \boldsymbol{\phi_\lambda(t)}$\\
		По теореме: $t = \sum\limits_\lambda \mathlarger{\frac{\lambda}{\boldsymbol{\phi'(\lambda)}}}\boldsymbol{\phi_\lambda(t)} = \sum\limits_\lambda \lambda p_\lambda$
	\end{proof}\ \\
	$\A \in End(V)\\
	\phi$ минималный многочлен, все корни $\in K (\Rightarrow \text{ все корни }\chi \in K \\
	\Rightarrow \text{т.е. все с.ч. }\in K \text{ -- I, II случаи})\\
	1 = \sum\limits_{\lambda} p_\lambda(t)$\\
	\begin{minipage}{0.2\textwidth}
		$\p_\lambda:=p_\lambda(\A)\\
		\p_\lambda\in End(V)\\
		\p_\lambda$ -- проекторы $?$
	\end{minipage}
	\begin{minipage}{0.6\textwidth}
		$
		\boxed{\E = \sum\limits_\lambda \p_\lambda}$ операторное разложение единицы\\
		$\uparrow$ это уже есть
	\end{minipage}\\\\
	Достаточно проверить $\underset{\lambda\neq \mu}{\p_\lambda \cdot \p_\mu} = \0\\
	\p_\lambda = p_\lambda(\A) = \underset{\nwarrow \nearrow}{f_\lambda(\A)\cdot \phi_\lambda(\A)}\\
	\lambda\neq \mu\\
	\p_\mu = p_\mu(\A) = \underset{\nwarrow\nearrow}{f_\mu(\A)\cdot\phi_\mu(\A)}\\
	\text{перестановочны, т.к. многочлены от }\A\\\\
	\begin{array}{llr}
		\multicolumn{3}{l}{
			\p_\lambda\p_\mu = f_\lambda(\A)\cdot f_\mu(\A) \underset{\mathlarger{\uparrow}}
			{\phi_\lambda(\A)} \cdot \boldsymbol{\phi_\mu(\A)} = \0
		}\\
	\updownarrow & \text{содержит}&\\
	(p_\lambda \cdot p_\mu \vdots \phi \text{ см. замеч. 1}) & \eta(\A)\boldsymbol{(t-\mu)^{m(\mu)}} & \phi(\A) = \0
	\end{array}\\\\
	\Rightarrow \p_\lambda $ проекторы -- \textbf{спектральные проекторы $\A$}\\
	$Im\p_\lambda$ \textbf{спектральное подпространство}\\
	$\underset{7.5}{\Rightarrow} \boxed{V = \bigoplus\limits_\lambda Im\p_\lambda}$
	\begin{examples}
		$A = \begin{pmatrix}
			1 & -3 & 4\\
			4 & -7 & 8\\
			6 & -7 & 7
		\end{pmatrix} \begin{matrix}
			\lambda_1 = -1 & \alpha(\lambda_1) = 2\\
			\lambda_2 = 3 & \alpha(\lambda_2) = 1
		\end{matrix}\\
		V_{\lambda_1} = span\begin{pmatrix}
			1\\2\\1
		\end{pmatrix} \ \ \ \; \gamma(\lambda_1) = 1 < \alpha(\lambda_1) \Rightarrow \text{ не о.п.с.}\\
		V_{\lambda_2} = span\begin{pmatrix}
			1\\2\\2
		\end{pmatrix} \ \ \ \; \gamma(\lambda_2) = 1\\
		\chi(t) = -(t+1)^2(t-3) \quad \phi_{\lambda_1} = (t-3)\\
		\phi(t) = (t+1)^2(t-3) \quad \phi_{\lambda_2} = (t+1)^2\\
		1 = \sum\limits_\lambda p_\lambda = p_{\lambda_1} + p_{\lambda_2} = f_{\lambda_1}\phi_{\lambda_1} 
		+ f_{\lambda_2}\cdot\phi_{\lambda_2} =\\
		= f_{\lambda_1}(t-3) + f_{\lambda_2}(t+1)^2\\
		\text{Прав. дробь }\mathlarger{\frac{1}{\phi}} = \sum\limits_\lambda \underset{\text{Правильн.}}{\mathlarger{\frac{f_\lambda \cdot \phi_\lambda}{\phi}}} = \sum\limits_\lambda \underset{\text{Правильн. дробь}}{\mathlarger{\frac{f_\lambda}{(t-\lambda)^{m(\lambda)}}}}\\
		deg \ f_\lambda < m(\lambda)\\
		\mathlarger{
			\frac{1}{(t+1)^2(t-3)} = \underset{\text{простейшие}}{\frac{A_1}{t+1} + \frac{A_2}{(t+1)^2}} + 
			\frac{A_3}{t-3} = \frac{-\frac{1}{16}t - \frac{5}{16}}{(t+1)^2} + \frac{\frac{1}{15}}{t-3}
		}\\
		1 = \underbracket{(-\frac{1}{16}t - \frac{5}{16})\overbrace{(t-3)}^{\phi_\lambda}}_{p_{\lambda_1}} + 
		\underbracket{\frac{1}{15} \overbrace{(t+1)^2}^{\phi_{\lambda_2}}}_{p_{\lambda_2}}\\
		\p_1 = p_{\lambda_1}(A) = \begin{pmatrix}
			0 & 1 & -1\\
			-2 & 3 & -2\\
			-2 & 2& -1
		\end{pmatrix} \; p_1 + p_2 = E\\
		\p_2 = p_{\lambda_2}(A) = \begin{pmatrix}
			1 & -1 & 1\\
			2 & -2 & 2\\
			2 & -2 & 2
		\end{pmatrix}
		$
	\end{examples}
	\begin{remark}
		$\forall \lambda: m(\lambda) = 1$\\
		Из следствия теоремы Лагранжа $t = \sum\limits_\lambda \lambda p_\lambda\\
		\boxed{\A = \sum\limits_\lambda \lambda \p_\lambda}  \nearrow \; \quad 1 = \sum p_\lambda \; \; $ 
		спектральное разложение о.п.с.\\\\
		$\boxed{\A \text{ о.п.с.} \Leftrightarrow \forall \lambda: m(\lambda) = 1
		\text{     \ \ \ \ \; Доказательство позже}}$
	\end{remark}
	\begin{defin}
		$K_\lambda = Ker(\A - \lambda\E)^{m(\lambda)}$\\
		называется \textbf{корневым подпространством }$\A$
	\end{defin}
	\begin{theorem}\
		\begin{mylist}
			\item 
			$K_\lambda$ инвариантно относительно $\A$
			\item $Im \p_\lambda = K_\lambda$
			\item $(t-\lambda)^{m(\lambda)}$ минимальный многочлен $\A|_{K_\lambda = Im\p_\lambda}$
		\end{mylist}
		$\Rightarrow \boxed{V = \bigoplus\limits_\lambda K_\lambda}$
	\end{theorem}
	\begin{proof}\
		\begin{mylist}
			\item 
			$x\in K_\lambda \overset{?}{\Rightarrow} \A x \in K_\lambda\\
			\underset{\stackrel{\nwarrow\nearrow}{\text{перестановочны}}}{(\A-\lambda\E)^{m(\lambda)} \A x}
			= \A \underbracket{(\A-\lambda\E)^{m(\lambda)} x \in K_\lambda}_{=\0} = \0\\
			\Rightarrow
			 \A x \in Ker(\A - \lambda\E)^{m(\lambda)}$
			 \item 
			 $(\A - \lambda \E)^{m(\lambda)}\p_\lambda = (\A - \lambda\E)^{m(\lambda)} f_\lambda(\A)\phi_\lambda(\A) =\\
			 = f_\lambda(\A) \cdot \underbracket{(\A - \lambda \E)^{m(\lambda)} \phi_\lambda(\A)}_{\phi(\A)} = \0\\
			 \forall x \in V\\
			 (\A - \lambda \E)^{m(\lambda)} \underbracket{\p_\lambda x}_{\in Im\p_\lambda} = \0 \Rightarrow Im\p_\lambda \subseteq Ker(\A - \lambda\E)^{m(\lambda)} = K_\lambda$\\
			 \textbf{Обратно: } $K_\lambda \overset{?}{\subseteq} Im\p_\lambda\\
			 x\in K_\lambda\\
			 \mu \neq K_\lambda \; \p_\mu x = \underbracket{
			 	\underset{
			 		\stackrel{\nearrow}
			 		{
			 			\text{содержит }(\A-\lambda\E)^{m(\lambda)}
		 			}
		 		}
	 			{
	 				f_\mu(\A)\phi_\mu(\A)x
 				}
	 	 	}_{\eta(\A)(\A - \lambda\E)^{m(\lambda)}} = \eta(\A)\cdot \underbracket{(\A-\lambda\E)^{m(\lambda)}x\in K_\lambda}_{=\0} = \0\\
 	 		x = \E x = \sum\limits_\mu \underset{\stackrel{||}{\0 \ \mu\neq \lambda}}{\p_\mu x} = \p_\lambda x \in Im\p_\lambda \Rightarrow K_\lambda \subseteq Im\p_\lambda \\
 	 		\Rightarrow \boxed{K_\lambda = Im\p_\lambda}$\newpage
 	 		\item 
 	 		$(t-\lambda)^{m(\lambda)}$ минимальный многочлен для $\A|_{K_\lambda = Im\p_\lambda}$ $?\\
 	 		(\A - \lambda \E)^{m(\lambda)}$ аннулятор $\A|_{K_\lambda}$\\
 	 		Минимальный?\\\\
 	 		$\pu \overset{\psi}{\text{не минимальный}}\\
 	 		\psi_1 = (t-\lambda)^{m(\lambda) - 1} \; \pu \text{ это минимальный многочлен}\\
 	 		\phi_1:= (t-\lambda)^{m(\lambda) - 1}\phi_\lambda(t) = $ аннулятор $\A?\\
 	 		\underset{\mu \neq \lambda}{\phi_1(\A) \p_\mu} = (\A-\lambda\E)^{m(\lambda) - 1} \phi_\lambda(\A) f_\mu (\A) \phi_\mu (\A) = \\
 	 		= \ldots \phi_\lambda(\A) \phi_\mu(\A) = \eta(\A) \cdot \phi(\A) = \0\\
 	 		\forall x \ \phi_1(\A) \p_{\lambda}x = (\A-\lambda\E)^{m(\lambda) - 1}\phi_\lambda(\A) \p_\lambda x = \\
 	 		= \phi_\lambda(\A) \underbracket{\underbracket{(\A - \lambda\E)^{m(\lambda) - 1}}_{\psi_1(\A)}
 	 		\underbracket{\p_\lambda x}_{\in Im\p_\lambda = K_\lambda}
  		   	}_{\underset{\text{ мин. многочлен по предположению}}{\psi_1(\A|_{K_\lambda})x}} = \0\\
  	   		\phi_1(\A)\p_\lambda = \0\\
  	   		\phi_1(\A)\cdot\E = \underbracket{\phi_1(\A) \sum\limits_\mu \p_\mu}_{\phi_1(\A)\p_\lambda + \sum\limits_{\mu\neq \lambda}\phi_1(\A) \p_\mu} = \0\\
  	   		\Rightarrow \phi_1 \text{ аннулятор }\A$, но степени $< \phi\\
  	   		deg\ \phi_1 = m-1 \Rightarrow \text{противоречие мин. }\phi \Rightarrow (t-\lambda)^{m(\lambda)}$ минимальный мн-н $\A|_{K_\lambda}$
		\end{mylist}
	\end{proof}
	\begin{corollary}
		$A$ о.п.с. $\Leftrightarrow \forall \lambda: m(\lambda) = 1$
	\end{corollary}
	\begin{proof}
		$(\Rightarrow) \ \; \; \A$ о.п.с.\\
		$\phi(t) \ \prod\limits_\lambda (t-\lambda) \; \; \; $ покажем что это минимальный многочлен $\A\\
		V = \bigoplus\limits_\lambda V_\lambda$ -- собственные подпространства $\A\\
		\forall v \in V \; \exists! \ v = \sum\limits_\lambda v_\lambda, v_\lambda \in V_\lambda\\
		\phi(\A) v = \prod\limits_\lambda (\A - \lambda\E) \sum\limits_\mu v_\mu = \\
		= \sum\limits_\mu \underbracket{\prod\limits_\lambda (\A - \lambda \E)}_{\phi_\mu (\A) \cdot (\A - \mu \E)} v_\mu = \sum\limits_\mu \phi_\mu (\A)\underbracket{(\A-\mu\E)v_\mu}_{\stackrel{||}{\0}} = \0\\
		v_\mu \in V_\mu = Ker(\A-\mu\E) \nearrow\\
		\Rightarrow \phi \text{ аннулятор }\A \Rightarrow$ очевидно минимальная степень $\Rightarrow$ минимальный многочлен.\\
		$(\Leftarrow) \forall \lambda: m(\lambda) = 1\\
		\underset{\stackrel{||}{Im\p_\lambda}}{K_\lambda} = Ker(\A - \lambda\E)^1 = V_\lambda\\
		V = \bigoplus\limits_\lambda K_\lambda = \bigoplus\limits_\lambda V_\lambda \Leftrightarrow \A $ о.п.с.
	\end{proof}
	\begin{examples}\ \\
		$Im\p_1 = Ker(A - \lambda_1 E)^2 = K_{\lambda_1}\\
		Im \p_2 = Ker(A - \lambda_2 E)^2 = K_{\lambda_2}\; \; \; \quad$ --- упр.
	\end{examples}
\end{document}