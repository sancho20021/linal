\documentclass[../main.tex]{subfiles}
\begin{document}
	\subsection{Операторное разложение единицы. Корневые подпространства.}
	\begin{minipage}{0.5\textwidth}
		$\phi(t) = \prod\limits_\lambda (t-\lambda)^{m(\lambda)}$
	\end{minipage}
	\begin{minipage}{0.5\textwidth}
		$\sum\limits_\lambda m(\lambda) = m\\
		deg \ \phi = m$
	\end{minipage}\\
	$P_{m-1}$ -- линейное пространство многочленов степени не выше $m-1\\
	dimP_{m-1} = m\\
	\phi(t) = \underset{\overset{\mathlarger{\nwarrow \nearrow}}{\text{вз. просты}}}{(t-\lambda)^{m(\lambda)}\phi_\lambda(t)} \ \ \; 
	\begin{matrix}
	\phi_\lambda(t) = \prod\limits_{\mu\neq \lambda} (t-\mu)^{m(\mu)}\\
	\phi_\lambda(\lambda) \neq 0\\
	\phi_\lambda(\mu) = 0\\
	\mu \neq \lambda
	\end{matrix}$
	\begin{defin}
		$I_\lambda = \{p\in P_{m-1} | p\vdots \phi_\lambda\}$\\
		\textbf{Главный идеал}, порожденный многочленом $\phi_\lambda = \\
		= \{f\in P_{m(\lambda)-1}|p = f_\lambda \phi_\lambda  \}\\
		I_\lambda$ -- линейное подпространство $P_{m-1}\\
		p_{1, 2}\vdots \phi_\lambda \Rightarrow (p_1 + \alpha p_2)\vdots\phi_\lambda$
	\end{defin}
	\begin{theorem}
		$P_{m-1} = \bigoplus\limits_\lambda I_\lambda$
	\end{theorem}
	\begin{proof}\
		\begin{mylist}
			\item Дизъюнктность.\\
			$\0 = \sum\limits_\lambda \underbrace{f_\lambda\phi_\lambda}_{\in I_\lambda}  = 
			f_\lambda\cdot \phi_\lambda + \underbrace{\sum\limits_{\mu\neq \lambda}f_\mu \underbrace{\phi_\mu}_{\vdots (t-\lambda)^{m(\lambda)}}}_{\vdots (t-\lambda)^{m(\lambda)}}\\
			\Rightarrow f_\lambda \cdot \underset{\text{вз. просты}}{\phi_\lambda \vdots (t-\lambda)^{m(\lambda)}} \Rightarrow \underset{\stackrel{\uparrow}{deg \ f_\lambda = m(\lambda)-1}}{f_\lambda} \vdots (t-\lambda)^{m(\lambda)} \Rightarrow f_\lambda \equiv 0 \\
			\Rightarrow \forall \lambda \ \ \ f_\lambda \equiv 0 \Rightarrow f_\lambda \phi_\lambda \equiv \0 \Rightarrow \text{Дизъюнктны}$
			\item
			\belowbaseline[-12pt]{
				$\begin{array}{lcl}
				dim P_{m-1} & = & m\\
				|| & & \\
				\sum\limits_\lambda dim I_\lambda & = & \sum\limits_\lambda m(\lambda) = m
				\end{array}$}\\\\
			$I_\lambda \subset P_{m-1}\\
			\\
			\Rightarrow P_{m-1} = \bigoplus\limits_\lambda I_\lambda$
		\end{mylist}
	\end{proof}
	\begin{corollary}
		$\forall p \in P_{m-1} \ \exists! \ p = \sum\limits_\lambda p_\lambda\\
		p_\lambda \in I_\lambda\\
		\boxed{1 = \sum\limits_\lambda p_\lambda\text{ -- полиномиальное разложение единицы}}$ 
	\end{corollary}
	\newpage
	\begin{remark} \
		\begin{mylist}
			\item 
			$\lambda \neq \mu\\
			\begin{array}{rclcl}
			p_\lambda & \cdot & p_\mu & \vdots & \phi\\
			|| & & || & &\\
			f_\lambda\phi_\lambda & & f_\mu\phi_\lambda & = & \eta \cdot \phi\\
			& & \uparrow & & \\
			& & (t-\lambda)^{m(\lambda)} & &
			\end{array}$
			\item $\forall \lambda \ m(\lambda) = 1$\\
			\textbf{Если.} Т. е. все корни $\phi$ взаимно простые.\\
			$f_\lambda = const \ \ \ (def \ f_\lambda = m(\lambda) - 1 = 0)$
		\end{mylist}
	\end{remark}
	
\end{document}