\documentclass[../main.tex]{subfiles}

\begin{document}
	\subsection{Инварианты линейного отображения}
	\textbf{Инвариант - свойство, которое сохраняется при некоторых определенных преобразованиях}\\
	$\mathscr{v} = \A\mathscr{u} \leftrightarrow v = Au$\\
	Форма записи действия линейного отображения на вектор инвариантна относительно замены базиса.\\
	$v' = A'u'$
	\begin{defin}
		$A_{m\times n}$\\
		$ImA = span(A_1, A_2,\ldots A_n) = \{\sum\limits_{i = 1}^n \alpha_i A_i|\alpha_i\in K\} = \\
		\{y = Ax \in \R^m(\mathbb{C}^m)| x\in \R^n(\mathbb{C}^n)\}$\\
		$x = \begin{pmatrix}
		\alpha_1\\\vdots\\\alpha_n
		\end{pmatrix}\\
		rgA = dimImA$ --- ранг матрицы\\
		$KerA = \{x\in\R^n(\mathbb{C}^n)|Ax = \0 \} =$ \{множество решений СЛОУ \} --- ядро матрицы\\
		$dimKerA = n-rgA = defA$ --- дефект матрицы\\
		$\boxed{rgA + defA = n}$ --- аналогично теореме о ранге и дефекте
	\end{defin}
	\begin{theorem}
		$\forall \A \in L(U, V)$\\
		\fbox{
			\parbox{73px}{
				$rg\A = rg A$\\
				$def\A = def A$
			}
		}, \\
		где матрица $A$ -- матрица линейного отображения в некоторых базисах пространств $U$ и $V$.\\
		$rg\A$, $def\A$ инвариантны относительно выбора базиса.
	\end{theorem}
	\begin{proof}
		$\A\leftrightarrow\underset{(\xi, \eta)}{A} \xi = (\xi_1\ldots\xi_n)$ базис $U$\\
		$\eta = (\eta_1\ldots\eta_m)$ базис $V$\\
		$Im\A = span(\A\xi_1\ldots\A\xi_n)\\
		\A\xi_i$\stackanchor{$\leftrightarrow$}{$\cong$}$A_i$\\
		Координатный изоморфизм.\\
		Пусть $rgA = k \Rightarrow k$ столбцов линейно независимы, а остальные -- их линейная комбинация.\\
		По свойствам изоморфизма это означает, то из $\A\xi_1\ldots\A\xi_n \ \  k$ линейно независимые, а остальные -- их линейная комбинация $\Rightarrow rg\A = dimIm\A = k\\$
		\belowbaseline[-10pt]{$\begin{matrix}
			dimU\\
			\parallel\\
			n
			\end{matrix}$
		} = \belowbaseline[-10pt]{
			$\begin{matrix}
			rg\A\\
			\parallel\\
			rgA\\
			\parallel\\
			k
			\end{matrix}$
		} $+ def\A\\
		def\A = n-rgA = n-k = dim$ пространства решений $Ax=0 = defA$ 
	\end{proof}
	\begin{corollary}
		$\A$ изоморфизм $\Leftrightarrow A$ невырожденная ($\exists A^{-1}$), где $A$ матрица в некотором базисе.
	\end{corollary}
	\begin{proof}
		Изоморфизм $\Leftrightarrow$ \stackanchor{$defA = 0$}{$dimU = dimV$} $\Leftrightarrow rgA = n \Leftrightarrow A$ невырожденная. 
	\end{proof}
	\begin{theorem}
		$det\A$ не зависит от выбора базиса пространства $V$ (т.е. является инвариантом относительно выбора базиса). И при этом $det\A = detA$, где $A$ -- матрица оператора $\A$ в некотором базисе.
	\end{theorem}
	\begin{proof}
		$V \ e_1\ldots e_n\\
		det\A = det(\A e_1\ldots\A e_n)\\
		\A e_k = \sum\limits_{i_k=1}^n a_{i_k k}e_{i_k} \xrightarrow{A = (a_{ij})} A_k = \begin{pmatrix}
		a_{1k}\\\vdots\\a_{nk}
		\end{pmatrix} = $
		($det$ $n$-форма, т. е. полиномиальная форма) \\
		$ = \sum\limits_{i_1 = 1}^n\sum\limits_{i_2 = 2}^n\ldots\sum\limits_{i_n = n}^n a_{i_1 1}a_{i_2 2}\ldots a_{i_n n}
		\ det(e_{i_1}, e_{i_2}\ldots e_{i_n}) = $ 
		($n$-форма -- 2 одинаковых аргумента $\Rightarrow det = 0$)$\\
		= \sum\limits_{\sigma = (i_1\ldots i_n)} a_{i_1 1} a_{i_2 2}\ldots a_{i_n n} 
		\overbracket{det(\underset{\text{все разные}}{e_{i_1}\ldots e_{i_n}})}^{(-1)^{\E(\sigma)} \ \ det(e_1\ldots e_n) = 1}
		= \sum\limits_{\sigma = (i_1\ldots i_n)}(-1)^{\E(\sigma)} a_{i_1 1} a_{i_2 2}\ldots a_{i_n n} = det A$\\\\
		$e'_1\ldots e'_n $ базис $V\\
		T = T_{e\rightarrow e'}\\
		det\A = det A' \overset{?}{=} det A\\
		A' = T^{-1} A T\\
		det A' = det T^{-1} \cdot detA \cdot det T = det A$
	\end{proof}
	\begin{defin}
		$A, B$ называются подобными, если \\
		$\exists $ невырожденная $C: B = C^{-1} A C$
	\end{defin}
	\begin{examples}
		Матрицы линейного оператора в разных базисах подобны\\
		$A' = T^{-1}AT$\\
		$A, B$ подобны $\Rightarrow det A = det B$
	\end{examples}
	\begin{corollary}
		$f$ -- $n$-форма на $V\\
		\forall \xi_1\ldots\xi_n \ \ \forall \A \in End(V)\\
		\Rightarrow \boxed{f(\A\xi_1\ldots\A\xi_n) = det\A \ f(\xi_1\ldots\xi_n)}$
	\end{corollary}
	\begin{proof}
		$f(\A\xi_1\ldots\A\xi_n) = \\
		\underset{n\text{-форма}}{
			g(\xi_1\ldots\xi_n)} = det(\xi_1\ldots\xi_n)\cdot g(e_1\ldots e_n) = \\
		det(\xi_1\ldots\xi_n)\cdot \underset{\text{смотри док-во теоремы}}{f(\A e_1\ldots \A e_n)} = 
		det(\xi_1\ldots\xi_n)\sum\limits_\sigma (-1)^{\E(\sigma)} a_{i_1 1}\ldots a_{i_n n}\cdot f(e_1\ldots e_n) = \\
		\A e_k = \sum\limits_{k = 1}^n a_{i_k k}e_{i_k} =
		\underbracket{det(\xi_1\ldots\xi_n)f(e_1\ldots e_n)}_{f(\xi_1\ldots\xi_n)}\underbracket{det A}_{det\A}
		$
	\end{proof}
	\begin{remark}
		$A$ -- линейный оператор, $B_{n\times n}\\
		AB = (AB_1 \ AB_2 \ldots AB_n)\\
		det(AB) = det(AB_1 \ldots AB_n) = \\
		= detA\cdot det(B_1\ldots B_n) = detA \cdot det B$
	\end{remark}
	\begin{corollary}
		$\A, \B \in End(V)\\
		det(\A\B) = det\A \cdot \B$
	\end{corollary}
	\begin{proof}
		$det(\A\B) = det(AB) = detA\cdot detB = det\A \cdot det\B$
	\end{proof}
	\begin{corollary}
		$\A\in Aut(V)\\
		\Leftrightarrow det\A \neq 0$\\
		Причем $det\det\A^{-1} = \frac{1}{det\A}$
	\end{corollary}
	\begin{proof}
		Из следствия 2\\
		$\A\A^{-1} = \A^{-1}\A = \E\\
		det\A \cdot det\A^{-1} = det\E = 1 \Rightarrow \ldots$
	\end{proof}
	\begin{examples}$V_3$\\
		\begin{multicols}{2}
			\includegraphics[width=150px]{7}\\	
			$V_{abc \text{--правая тройка}} = \underset{\text{смешанное пр-е}}{\vec{a}\vec{b}\vec{c}} = 
			f(\underset{\text{3-форма}}{\vec{a}\vec{b}\vec{c}})$\\
			$\A\in End(V_3) \ u\in V_3 \rightarrow v = \A u \in V_3$\\
			Как поменяется объем параллелепипеда при линейном преобразовании? 
		\end{multicols}
		$\A (V_{(\vec{a}\vec{b}\vec{c})}) = f(\A\vec{a}, \A\vec{b}, \A\vec{c}) = 
		det\A\cdot f(\vec{a}, \vec{b}, \vec{c}) = det\A \cdot V(\vec{a} \vec{b} \vec{c})\\
		\lambda = |det\A|\ \ \ \ $ Объем увеличится в $\lambda$ раз.\\
		\begin{mylist}
			\item 
			$\A: V_3 \rightarrow V_3$\\
			Оператор подобия\\
			$\forall u\in V_3: \A u = \mu u, \mu \in \R$\\
			\begin{minipage}{0.2\textwidth}
				$A?$
			\end{minipage}
			\begin{minipage}{0.4\textwidth}
				$\A\vec{i} = \mu\vec{i} \leftrightarrow \begin{pmatrix}\mu\\0\\0\end{pmatrix}\\
				\A\vec{j} = \mu\vec{j} \leftrightarrow \begin{pmatrix}0\\\mu\\0\end{pmatrix}\\
				\A\vec{k} = \mu\vec{k} \leftrightarrow \begin{pmatrix}
				0\\0\\\mu
				\end{pmatrix}$
			\end{minipage}
			\begin{minipage}{0.4\textwidth}
				$A = \begin{pmatrix}
				\mu & 0 & 0\\
				0 & \mu & 0\\
				0 & 0 & \mu
				\end{pmatrix}$
			\end{minipage}
			$\lambda = |det\A| = |detA| = |\mu^3|$\newpage
			\item 
			$\A: V_3\rightarrow V_3$\\
			\textbf{Оператор поворота}\\\\
			$\A:  \ \ \ \ \ \begin{matrix}
			\vec{i} \rightarrow e_1 \nearrow \\\vec{j}\rightarrow e_2 \rightarrow\\\vec{k}\rightarrow e_3\searrow
			\end{matrix} \ \ \ 
			\begin{matrix}
			\begin{pmatrix}
			\cos\alpha_1\\
			\cos\beta_1\\
			\cos\gamma_1
			\end{pmatrix}\\
			\begin{pmatrix}
			\cos\alpha_2\\
			\cos\beta_2\\
			\cos\gamma_2
			\end{pmatrix}\\
			\begin{pmatrix}
			\cos\alpha_3\\
			\cos\beta_3\\
			\cos\gamma_3
			\end{pmatrix}\\
			\end{matrix}$\\
			\begin{minipage}{0.5\textwidth}
				\includegraphics[width=0.5\textwidth]{8}
			\end{minipage}
			\begin{minipage}{0.5\textwidth}
				$|e_i| = 1\\
				(e_i, e_j) = 0\\
				i\neq j$
			\end{minipage}\\\\
			"$\A(V_{\vec{a}\vec{b}\vec{c}})$" $= det\A\cdot V_{\vec{a}\vec{b}\vec{c}} = V_{\vec{a}\vec{b}\vec{c}}\\\\
			A = \begin{pmatrix}
			\cos\alpha_1 & \cos\alpha_2 & \cos\alpha_3\\
			\cos\beta_1 & \cos\beta_2 & \cos\beta_3\\
			\cos\gamma_1 & \cos\gamma_2 & \cos\gamma_3\\
			\end{pmatrix}\\\\
			detA = |\cdots| \underset{\text{Смешанное произведение}}{e_1e_2e_3} = 1$\\
			$(detA)^2 = detA\cdot detA^T = det(AA^T) = det\begin{pmatrix}
			(e_1, e_1) & (e_1, e_2) & (e_1, e_3)\\
			(e_2, e_1) & (e_2, e_2) & (e_2, e_3)\\
			(e_3, e_1) & (e_3, e_2) & (e_3, e_3)\\
			\end{pmatrix}
			= detE = 1\\
			|detA| = 1$
		\end{mylist}
	\end{examples}
	\begin{stat}
		$A, B$ подобные матрицы $\Rightarrow trA = trB$\\
		$trace = $ след
	\end{stat}
	\begin{proof}
		$A, B$ подобные $\Rightarrow\\
		\exists\ C$ невырожденная$: C^{-1}(AC) = B$\\
		$$trB = \sum\limits_{i=1}^n b_{ii} = \sum\limits_{i = 1}^n \sum\limits_{j=1}^n C^{\text{''}-1\text{''}}_{ij}(AC){ji} = 
		\sum\limits_{i = 1}^n \sum\limits_{j=1}^n \sum\limits_{k = 1} ^ n C^{''-1''}_{ij} a_{jk} C_{ki} = 
		\sum_{j=1}^n\sum_{k=1}^n a_{jk} \underbracket{\sum_{i=1}^n C_{ki} C^{''-1''}_{ij}}_{\delta_{kj}} = \sum_{k=1}^n a_{kk} = trA 
		$$
		$\boxed{\delta_{kj} = \left[\begin{matrix}1, k=j\\0, k\neq j \end{matrix} \right.} \ \ CC^{-1} = E$
	\end{proof}
	\begin{defin}
		$tr\A = trA$, где $A$ -- матрица оператора в некотором базисе.\\
		$tr\A = trA = trA'$ --- не зависит от выбора базиса, т.к. $A$ и $A'$ подобны.
	\end{defin}
	\begin{defin}
		$L\subset V \ L$ инвариантно относительно $\A\in End(V)$
		если $\forall u\in L: \A u\in L$
	\end{defin}
	\begin{examples}\ \\
		\begin{mylist}
			\item $\0, V$ инвариантны относительно $\A$
			\item $Ker\A, Im\A $ инвариантны относительно $\A$\\
			\begin{minipage}{0.3\textwidth}
				\includegraphics{9}
			\end{minipage}
			\begin{minipage}{0.7\textwidth}
				\begin{flushleft}
					$\A:V_3\rightarrow V_3$\\
					Поворот вектора(пр-ва) относительно оси $l$ на угол $\alpha$
				\end{flushleft}
			\end{minipage}\\
			\begin{minipage}{0.3\textwidth}
				\includegraphics{10}
			\end{minipage}
			\begin{minipage}{0.7\textwidth}
				\begin{flushleft}
					Плоскость $\perp l$ инвариантна относительно $\A$\\
					$P = x_0 + L$ инвариантно
				\end{flushleft}
			\end{minipage}
		\end{mylist}
	\end{examples}
	\begin{theorem}
		$L\subset B \ \ \A\in End(V)$. Линейное пространство инвариантно относительно $\A$\\
		$\Rightarrow \exists$ базис пространства $V$, т.ч. матрица оператора $\A$ в этом базисе\\ будет иметь вид: $A = \left(\begin{array}{c|c}A_1 & A_2\\
		\hline
		\0 & A_3\end{array}\right)\\
		A_1 k\times k$ где $k = dimL$
	\end{theorem}
	\begin{proof}
		$L = span(\underset{\text{базис}}{e_1\ldots e_k})$\\
		Дополним до базиса $V: e_1\ldots e_k e_{k+1}\ldots e_n\\
		e_i \in L \Rightarrow \underset{1\leq i \leq k}{\A e_i}\in L = \sum\limits_{m=1}^k a_{mi}e_m + 
		\sum\limits_{m = k+1}^n 0\cdot e_m
		\leftrightarrow A_i^1 = \begin{pmatrix}
		a_{1i}\\\vdots\\ a_{ki}\\0\\\vdots\\0
		\end{pmatrix}\\
		\underset{k+1\leq i \leq n}{\A e_i} = \sum\limits_{j=1}^n a_{ij} e_j \leftrightarrow 
		A^{2, 3}_i = \begin{pmatrix}
		a_{1i}\\\vdots\\a_{ni}
		\end{pmatrix}
		\Rightarrow A = \begin{pmatrix}
		\boxed{\begin{matrix}
			a_{1i} & \\\vdots & A_i^1 \\a_{ki} & 
			\end{matrix}
		} & \boxed{
			\begin{matrix}
			A^{2, 3}_i\\\ldots
			\end{matrix}
		}\\
		0 & \boxed{\begin{matrix}
			\ldots\\\ldots
			\end{matrix}}
		\end{pmatrix}$
	\end{proof}
	\begin{corollary}
		$V = \bigoplus\limits_{i=1}^m L_i \ \ \ L_i$ инвариантно $\A\\
		\Rightarrow \exists \ $ базис пр-ва $V$, в котором матрица оператора $\A$ будет иметь \underline{блочно-диагональный вид:}\\
		$A = \begin{pmatrix}
		\boxed{A^1} & \ldots & 0\\
		& \boxed{A^2} & \\
		0 & & \boxed{A^n}
		\end{pmatrix}\\
		\left(\underset{\text{размерность матрицы}}{A^i}\right) = dimL_i$
	\end{corollary}
	\begin{proof}
		$L_1 = span(\underset{\text{базис}}{e^1_i\ldots e^{i_k}_i})$\\
		т.к. $\bigoplus$, то базис $V$ -- объединение базисов $L_i\\
		V = span(e^1_1\ldots e^{i_m}_m)\\
		\A^j e_i \in L_i \Rightarrow$ раскладываем по базису $L_i \Rightarrow$\\ на остальных позициях в столбике матрицы оператора будут нули.\\
		$A = \left(
		\begin{array}{c|c|c}
		\overset{\stackrel{L_1}{\underline{1\ldots i_1}}}{\begin{matrix}* & * &*\\ *& * & *\\
			* & * & *\end{matrix}} & 
		\overset{\stackrel{L_2}{\underline{i_1 + 1\ldots i_2}}}
		{\begin{matrix}0 & 0 & 0\\ 0 & 0& 0\\0&0&0\end{matrix}} &
		\begin{matrix}
		0\\0\\0
		\end{matrix} \\
		\hline 
		\begin{matrix}0 & 0 & 0\\ 0 & 0& 0\\0&0&0\end{matrix} 
		& \begin{matrix}* & * &*\\ *& * & *\\
		* & * & *\end{matrix} &
		\left.\begin{matrix}*\\\vdots\\ *\end{matrix}\right\}\\
		\hline
		\begin{matrix}0 & 0 & 0\\ 0 & 0& 0\\0&0&0 \end{matrix}& \begin{matrix}
		0 & 0 & 0\\ 0 & 0& 0\\0&0&0
		\end{matrix} & \begin{matrix}0\\0\\0\end{matrix}
		\end{array}\right) \text{отвечает позиции базисных элементов 
			пр-ва }L_i\text{в базисе }V$
	\end{proof}
	\begin{corollary}
		$V = \bigoplus\limits_{i=1}^m L_i \ \ \ \ L_i$ инвариантно относительно $\A\\
		\A\in End(V)
		\Rightarrow V = \bigoplus\limits_{i=1}^m Im \A|_{L_i}$
	\end{corollary}\newpage
	\begin{proof}
		$V = \bigoplus\limits_{i=1}^m L_i \Rightarrow \forall\ u \in V \ \  \exists! u = \sum\limits_{i = 1}^m u_i \in L_i\\
		Im\A \subset \sum\limits_{i = 1}^m Im\A|_{L_i}\\
		v\in Im\A = \A u = \sum\limits_{i = 1}^m \A u_i \in Im\A|_{L_i}\\
		$
		\textbf{Верно и "$\supset$"}\\
		Пусть $v_i \in Im\A|_{L_i} : v_i = \A u_i, u_i\in L_i\\
		\sum\limits_{i = 1}^m v_i = \sum\limits_{i = 1}^m \A u_i = \A(\sum\limits_{i = 1}^m u_i \in V)\in Im\A\\
		Im\A = \sum\limits_{i = 1}^m Im\A|_{L_i}\\
		\bigoplus$ прямая?\\
		$v_i\in Im\A|_{L_i}\\
		v_i = \A u_i \ \ \ \ u_i \in L_i\\
		\sum\limits_{i = 1}^m v_i = \0 \longleftarrow\\
		\text{Т.к. }L_i $ инвариантна $\Rightarrow \A u_i \in L_i \Rightarrow v_i \in L_i$, но $L_i$ дизъюнктны $\nwarrow \Rightarrow \forall i : v_i = \0\\
		\Rightarrow Im \A|_{L_i}$ дизъюнктны $\Rightarrow \bigoplus$
	\end{proof}
\end{document}